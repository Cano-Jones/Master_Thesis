\section{The Effective Action $W$}
For simplicity we consider a scalar field described by the action presented at equation \ref{eq: General scalar field action} (...)

According to the path integral formulation of quantum field theory, the time-ordered vacuum expectation value of $T_{\mu\nu}$ is given by
\begin{equation}
	\langle T_{\mu\nu}\rangle =\frac{\int \mathcal{D}[\phi]\,T_{\mu\nu}\,e^{iS_\text{M}[\phi]\hbar^{-1}}}{\int \mathcal{D}[\phi]\,e^{iS_\text{M}[\phi]\hbar^{-1}}}\equiv \frac{-2}{\sqrt{-g}}\frac{\delta W}{\delta g^{\mu\nu}},
\end{equation}
where the last definition mimics the energy momentum tensor describe in equation \ref{eq: Hilbert energy-momentum tensor}, where the quantity $W$ is said to be the effective action of the field for a semi-classic treatment.

 Now, would be useful to have a closed expression for $W$, this will be done using the generating functional 
$Z[J]\equiv \int \mathcal{D}[\phi]\,\exp{\{iS_\text{M}[\phi]\hbar^{-1}+i\int\td^4J(x)\phi(x)\}}$, where $J(x)$ is some external current, that will be consider zero for the following treatment. Substituting the definition of $T_{\mu\nu}$ (again, expression \ref{eq: Hilbert energy-momentum tensor}) on the previous equation, one finds the following relation
\begin{equation}
	\langle T_{\mu\nu}\rangle=\frac{-2}{Z[0]\sqrt{-g}}\int \mathcal{D}[\phi]\,\frac{\delta S_\text{M}[\phi]}{\delta g^{\mu\nu}}\,e^{iS_\text{M}[\phi]\hbar^{-1}}=\frac{2i\hbar}{Z[0]\sqrt{-g}}\frac{\delta Z[0]}{\delta g^{\mu\nu}},
\end{equation}
meaning that the effective action can be written as
\begin{equation}\label{eq: Effective Action Z[0]}
	W=-i\hbar \ln\left( \frac{Z[0]}{\mu_0\hbar^{\sfrac{1}{2}}}\right)+\text{C};
\end{equation}
where we have introduced an integral constant C, and a constant $\mu_0$ with the same units as $\mu$ (that is, inverse of length), in order to maintain the argument of the logarithm dimensionless.

The expression for $W$ can be simplified even further in terms of (supposedly) known parameters; to obtain such representation, consider that the action $S_\text{M}[\phi]$ given by equation \ref{eq: General scalar field action} can be written as
\begin{equation}
	S_\text{M}[\phi]=\frac{1}{2}\int\partial_\nu\!\left[\sqrt{-g}\,\phi\,\partial^\nu\!\phi\right]\td^4x-\int\frac{1}{2}\phi\left[\partial_\nu\partial^\nu+\mu^2+\xi R\right]\phi\,\sqrt{-g}\,\td^4x,
\end{equation}
where the first term is a total derivative and thus can be dropped. Next we could use the Dirac delta function to write
\begin{equation}
	\phi(x)=\int\phi(y)\frac{\delta^4(x-y)}{\sqrt{-g(x)}}\sqrt{-g(y)}\,\td^4 y;
\end{equation}
meaning that the action might be expressed as
\begin{equation}
	S_\text{M}[\phi]=-\frac{1}{2}\int\sqrt{-g(x)}\,\td^4x\int\phi(x)K(x,y)\phi(y)\sqrt{-g(y)}\,\td^4 y;
\end{equation}
where we defined the function
$K(x,y)\equiv \left[\partial_\nu\partial^\nu+\mu^2+\xi R(x)\right]\delta^4(x-y)\sfrac{}{\sqrt{-g(x)}}$. But what exactly represents $K(x,y)$? Well, considering the definition of an inverse function
\begin{equation}
	\int K(x,y)K^{-1}(y,z)\,\sqrt{-g(y)}\,\td^4y=\frac{\delta^4(x-z)}{\sqrt{-g(z)}},
\end{equation}
one ends up with the following relation
\begin{equation}
	\left[\partial_\nu\partial^\nu+\mu^2+\xi R(x)\right]K^{-1}(x,z)=\frac{\delta^4(x-z)}{\sqrt{-g(z)}};
\end{equation}
which is the definition of the Feynman propagator $G_F(x,z)$, meaning that
\begin{equation}
	K(x,y)=-G_F^{-1}(x,y).
\end{equation}
Using this relation, we can write the action as
\begin{equation}
	S_\text{M}[\phi]=\frac{1}{2}\int\sqrt{-g(x)}\,\td^4x\int\phi(x)G_F^{-1}(x,y)\phi(y)\sqrt{-g(y)}\,\td^4 y,
\end{equation}
which can be interpreted as the product of matrices of continuous index $\phi^\dagger G_F\,\phi$ (where $G_F$ is the operator related to the propagator $G_F(x,y)$ through $G_F(x,y)=\langle x|G_F|y\rangle$), and thus, one can compute the value of $Z[0]$ as
\begin{equation}
	Z[0]=\int\mathcal{D}[\phi]\,\exp{\left\{i\frac{\phi^\dagger G_F^{-1}\phi}{2\hbar}\right\}}\propto\left[\hbar^{-1}\,\text{det}\left(-G_F\right)\right]^{-\sfrac{1}{2}}.
\end{equation}
Substituting this value in equation \ref{eq: Effective Action Z[0]}, and by appropriately choosing the value of the constant C to compensate for the proportionality factor \footnote{Nevertheless, if another choice were to be made, the sum of both the constant C and the logarithm of the proportionality factor, is not a function of the metric, and thus, it is irrelevant under variations of $g_{\mu\nu}$.}, one can deduce the following expression for the effective action
\begin{equation}
	W=-\frac{i\hbar}{2}\ln\left[\mu_0^{-2} \,\text{det}\left(-G_F\right)\right]=-\frac{i\hbar}{2}\text{Tr}\left[\ln\left(-\mu_0^{-2}G_F\right)\right].
\end{equation}
Here we introduced the trace of the operator $-\mu_0^{-2}G_F$, this can be computed using the definition of the trace,
\begin{equation}
	\text{Tr}[A]\equiv \int A(x,x)\,\sqrt{-g(x)}\,\td^4x=\int \langle x|A|x\rangle\,\sqrt{-g}\,\td^4x;
\end{equation}
which will be relevant in what follows.
\subsection{Renormalization}
\begin{equation}
	G_F^{\text{DS}}(x,y)\equiv -i\frac{\Delta^{1/2}(x,y)}{(4\pi)^{\sfrac{(d+1)}{2}}}\int^\infty_0F(x,y;is)\exp{\left[-is\mu^2+\frac{\sigma(x,y)}{2is}\right]}(is)^{-\sfrac{(d+1)}{2}}\,\td (is)
\end{equation}
\begin{equation}
	G_F=-K^{-1}=-\int_0^\infty e^{-is K}\td (is)
\end{equation}
\begin{equation}
\left\langle x\left|e^{-isK}\right|y\right\rangle=i\frac{\Delta^{1/2}(x,y)}{(4\pi)^{\sfrac{(d+1)}{2}}}F(x,y;is)\exp{\left[-is\mu^2+\frac{\sigma(x,y)}{2is}\right]}(is)^{-\sfrac{(d+1)}{2}}
\end{equation}
\begin{equation}
	\left\langle x\left|(is)^{-1}e^{-isK}\right|y\right\rangle=i\int_{\mu^2}^\infty\frac{\Delta^{1/2}(x,y)}{(4\pi)^{\sfrac{(d+1)}{2}}}F(x,y;is)\exp{\left[-is\mu^2+\frac{\sigma(x,y)}{2is}\right]}(is)^{-\sfrac{(d+1)}{2}}\,\td \left(\mu^2\right)
\end{equation}
\begin{equation}
	\int_0^\infty(is)^{-1}e^{-is K}\td(is)=-\ln\left(\mu_0^2\,K\right)+\text{C}'=\ln\left(-\mu^{-2}_0\,G_F\right)+\text{C}'
\end{equation}
\begin{equation}
	W=\frac{i\hbar}{2}  \int_{\mu^2}^\infty \td \left(\mu^2\right)\int G_F(x,x)\,\,\sqrt{-g(x)}\,\td^4x
\end{equation}
\begin{equation}
	\mathcal{L}_{\text{eff}}\equiv \lim\limits_{y\to x}\frac{i\hbar c}{2}\int_{\mu^2}^\infty G_F^{\text{DS}}(x,y)\,\td\left(\mu^2\right)
\end{equation}
\begin{equation}
	\mathcal{L}_{\text{eff}}=\lim\limits_{y\to x}\frac{\hbar c}{2\mu_0^{d-3}}\frac{\Delta^{1/2}(x,y)}{(4\pi)^{\sfrac{(d+1)}{2}}}\int^\infty_0F(x,y;is)\exp{\left[-is\mu^2+\frac{\sigma(x,y)}{2is}\right]}(is)^{-\sfrac{(d+3)}{2}}\,\td (is)
\end{equation}
$\mu_0^{d-3}$ was introduced so that the units of $\mathcal{L}_{\text{eff}}$ remain to be the same as for $d=3$. Blablabla convergence for any $d$ and thus we take the limit

\begin{multline}
	\mathcal{L}_{\text{eff}}=\frac{\hbar c}{2\mu_0^{d-3}}\frac{1}{(4\pi)^{\sfrac{(d+1)}{2}}}\sum_{n=0}^\infty a_n(x)\int^\infty_0\exp{\left(-is\mu^2\right)}(is)^{n-\sfrac{(d+3)}{2}}\,\td (is)=\\
	=\frac{\hbar c}{2(4\pi)^{\sfrac{(d-1)}{2}}}\left(\frac{\mu}{\mu_0}\right)^{(d-3)}\sum_{n=0}^{\infty}a_n(x)\mu^{2(2-n)}\Gamma\left(n-\frac{d+1}{2}\right)
\end{multline}
\begin{subequations}
	\begin{gather}
		\Gamma\left(-\frac{d+1}{2}\right)=\frac{4}{d^2-1}\left(\frac{2}{3-d}-\gamma\right)+\mathcal{O}(d-3),\\
		\Gamma\left(1-\frac{d+1}{2}\right)=\frac{2}{1-d}\left(\frac{2}{3-d}-\gamma\right)+\mathcal{O}(d-3),\\
		\Gamma\left(2-\frac{d+1}{2}\right)=\frac{2}{3-d}-\gamma+\mathcal{O}(d-3).
	\end{gather}
\end{subequations}
\begin{equation}
	\left(\frac{\mu}{\mu_0}\right)^{d-3}=1+(d-3)\ln\left(\frac{\mu}{\mu_0}\right)+\mathcal{O}\left[(d-3)^2\right]
\end{equation}
\begin{equation}
	\mathcal{L}_{\text{eff}}^\infty=-\lim\limits_{d\to 3}\frac{\hbar c}{2(4\pi)^2}\left\{\frac{1}{d-3}+\frac{1}{2}\left[\gamma+2\ln\left(\frac{\mu}{\mu_0}\right)\right]\right\}\left[\mu^4a_0(x)-\mu^2a_1(x)+2a_2(x)\right]
\end{equation}
\begin{subequations}
	\begin{gather}
		a_0(x)=1,\quad a_1(x)=\left(\frac{1}{6}-\xi\right)R,\tag{\theequation \,\,a,b}\\
		a_2(x)=\frac{1}{180}\left(R_{\alpha\beta\gamma\sigma}R^{\alpha\beta\gamma\sigma}-R_{\alpha\beta}R^{\alpha\beta}\right)-\frac{1}{6}\left(\frac{1}{5}-\xi\right)\partial_\nu\partial^\nu R+\frac{1}{2}\left(\frac{1}{6}-\xi\right)^2R^2.\tag{\theequation \,\,c}
	\end{gather}
\end{subequations}
\begin{equation}
	\mathcal{L}_{\text{G}}'\equiv\mathcal{L}_{\text{G}}-\mathcal{L}_{\text{eff}}=\left(A+\frac{1}{2\kappa}\right)R-\left(B+\frac{1}{\kappa}\Lambda\right)-\lim\limits_{d\to 3}\frac{a_2(x)\hbar c}{(4\pi)^2}\left\{\frac{1}{d-3}+\frac{1}{2}\left[\gamma+2\ln\left(\frac{\mu}{\mu_0}\right)\right]\right\}
\end{equation}
\begin{subequations}
	\begin{gather}
		A\equiv \lim\limits_{d\to 3}\frac{\mu^2\,\hbar c}{2(4\pi)^2}\left(\frac{1}{6}-\xi\right)\left\{\frac{1}{d-3}+\frac{1}{2}\left[\gamma+2\ln\left(\frac{\mu}{\mu_0}\right)\right]\right\},\\
		B\equiv \lim\limits_{d\to 3}\frac{\mu^4\,\hbar c}{2(4\pi)^2}\left\{\frac{1}{d-3}+\frac{1}{2}\left[\gamma+2\ln\left(\frac{\mu}{\mu_0}\right)\right]\right\}
	\end{gather}
\end{subequations}
\begin{subequations}
	\begin{gather}
		G'\equiv G\left(1+2\kappa A\right)^{-1},\quad \Lambda'\equiv \left(\Lambda+B\kappa\right).
	\end{gather}
\end{subequations}
$S[g]=\int\frac{1}{2\kappa}f(R)\sqrt{-g}\,\td^4x$
\begin{equation}
	S[g,\phi]=\int\left[\frac{1}{2\kappa}f(R)-\mathcal{L}_{\text{eff}}\right]\sqrt{-g}\,\td^4x+ W_{\text{ren}}
\end{equation}
\begin{equation}
	W_{\text{ren}}=\int\left[\mathcal{L}_{\text{eff}}-\mathcal{L}_{\text{eff}}^{\infty}\right]\sqrt{-g}\td^4 x=-\frac{\hbar}{64\pi^2}\int\int_{0}^\infty\ln(is)\partial_{is}^3\left[F(x;is)e^{-is\mu^2}\right]\sqrt{-g}\,\td(is)\td^4x.
\end{equation}
\section{The Conformal Anomaly}
TBD