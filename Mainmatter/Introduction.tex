\begin{equation}
	\td l^2=g_{\mu\nu}\td x^\mu\td x^\nu
\end{equation}
\begin{equation}
	S=\int\left[\frac{1}{2\kappa}\left(R-2\Lambda\right)+\mathcal{L}_{\text{M}}\right]\sqrt{-g}\,\td^4 x
\end{equation}
$\kappa\equiv\frac{8\pi G}{c^4}$

Variation of $S$ with respect to the inverse metric ($g^{\mu\nu}$) gives
\begin{multline}
	\delta S=\int \left[\frac{\sqrt{-g}}{2\kappa}\frac{\delta R}{\delta g^{\mu\nu}}+\frac{R}{2\kappa}\frac{1}{\sqrt{-g}}\frac{\delta \sqrt{-g}}{\delta g^{\mu\nu}}-\frac{\Lambda}{\kappa}\frac{1}{\sqrt{-g}}\frac{\delta\sqrt{-g}}{\delta g^{\mu\nu}}+\frac{\delta\mathcal{L}_{\text{M}}}{\delta g^{\mu\nu}}+\frac{\mathcal{L}_{\text{M}}}{\sqrt{-g}}\frac{\delta\sqrt{-g}}{\delta g^{\mu\nu}}\right]\delta g^{\mu\nu}\,\sqrt{-g}\,\td^4x
\end{multline}
$\delta S=0$ and
\begin{equation}
	\frac{\delta R}{\delta g^{\mu\nu}}=R_{\mu\nu}\hspace{1.0cm}\frac{1}{\sqrt{-g}}\frac{\delta\sqrt{-g}}{\delta g^{\mu\nu}}=-\frac{1}{2}g_{\mu\nu}
\end{equation}
\begin{equation}
	R_{\mu\nu}-\frac{1}{2}g_{\mu\nu}R+\Lambda g_{\mu\nu}=-2\frac{8\pi G}{c^4}\left(\frac{\delta\mathcal{L}_{\text{M}}}{\delta g^{\mu\nu}}-\frac{1}{2}\mathcal{L}_{\text{M}}g_{\mu\nu}\right)
\end{equation}
\begin{equation}
	T_{\mu\nu}\equiv \mathcal{L}_{\text{M}}g_{\mu\nu}-\frac{\delta\mathcal{L}_{\text{M}}}{\delta g^{\mu\nu}}=\frac{-2}{\sqrt{-g}}\frac{\delta \left(\mathcal{L}_{\text{M}}\sqrt{-g}\right)}{\delta g^{\mu\nu}}
\end{equation}
\cite{Energy-MomentumTensor}
\begin{equation}
	\nabla_\mu T^{\mu\nu}=0
\end{equation}
(and its symmetric)
\section{Construction of covariant actions}
The equivalence principle says that field equations must be invariant with respect to local Lorentz transformations $\Lambda(x)$. To work on a local flat spacetime, we use the tetrad formalism
\begin{equation}
	g_{\mu\nu}=e^a_\mu e^b_\nu \eta_{ab}
\end{equation}
partial derivatives transform like
\begin{equation}
	\partial_\mu\to\frac{\partial x^\nu}{\partial y^\mu}\partial_\nu
\end{equation}
but, we see that $e^\mu_a\partial_\mu$ is not covariant, since
\begin{equation}
	e^\mu_a\partial^a\phi(x)\to \Lambda^b_ae^\mu_b\partial_\mu\left[\rho\left(\Lambda\right)\phi(x)\right]= \Lambda^b_ae^\mu_b\left[\rho\left(\Lambda\right)\partial_\mu\phi+\partial_\mu\rho\left(\Lambda\right)\,\phi\right]
\end{equation}
a covariant derivative $D_\mu$ should transform as
\begin{equation}
	D_a\phi\to\Lambda_a^b\rho\left(\Lambda\right)D_b\phi
\end{equation}
therefore we need to define a better derivative, a common option is
\begin{equation}
	D_a\equiv e^\mu_a\left(\partial_\mu+\Gamma_\mu\right)
\end{equation}
where, for the derivative to be covariant, the connection $\Gamma_\mu$ must transform as
\begin{equation}
	\Gamma_\mu\to \rho\left(\Lambda\right)\Gamma_\mu\rho^{-1}\left(\Gamma\right)-\left[\partial_\mu\rho\left(\Lambda\right)\right]\rho^{-1}\left(\Lambda\right)
\end{equation} 
such connection can be written as
\begin{equation}
	\Gamma_\mu=\frac{1}{2}\Sigma^{ab}e^\nu_a\nabla_\mu e_{b\nu}
\end{equation}
where $\Sigma^{ab}$ are the Lorentz generators and $e^\nu_a\nabla_\mu e_{b\nu}\equiv \omega_{ab\mu}$ is the torsion free spin connection.


Now, in order to obtain the equations of motion for the matter fields, consider the lagrangian
\begin{equation}
	\mathcal{L}_{\text{M}}=\mathcal{L}_{\text{M}}\big[\phi^\alpha(x),\,\nabla_\mu\phi^\alpha(x)\big]
\end{equation}
variations of $S$ with respect of $\phi^\alpha$ result in
\begin{equation}
	\delta S=\int\left[\frac{\partial \mathcal{L}_\text{M}}{\partial \phi^\alpha}\delta\phi^\alpha+\frac{\partial \mathcal{L}_\text{M}}{\partial\left(\nabla_\mu\phi^\alpha\right)}\nabla_\mu\left(\delta\phi^\alpha\right)\right]\sqrt{-g}\,\td^4x
\end{equation}
and thus, after applying the generalized Gauss Theorem on a curved background, and considering that field variations vanish at the boundaries, one obtains
\begin{equation}
	\frac{\partial \mathcal{L}_\text{M}}{\partial \phi^\alpha}-\nabla_\mu\left[\frac{\partial \mathcal{L}_\text{M}}{\partial\left(\nabla_\mu\phi^\alpha\right)}\right]=0
\end{equation}
\begin{equation}
	\Pi_\alpha\equiv \frac{\partial\mathcal{L}_\text{M}}{\partial\left(\nabla_0\phi^\alpha\right)}
\end{equation}
\section{Scalar field}
\begin{equation}
	S\big[\phi\big]=\int\frac{1}{2}\Big[\nabla_\nu\phi\,\nabla^\nu\phi-\mu^2\phi^2-\xi R\phi^2\Big]\sqrt{-g}\;\td^4 x
\end{equation}
equations of motion (Klein-Gordon)
\begin{equation}
	\big[\nabla_\nu\nabla^\nu-\mu^2-\xi R\big]\phi=0
\end{equation}
\begin{equation}
	T_{\mu\nu}=\nabla_\mu\phi\,\nabla_\nu\phi -\frac{1}{2}g_{\mu\nu}\left[\nabla^\sigma\phi\nabla_\sigma\phi-\mu^2\phi^2\right]+\xi\left[R_{\mu\nu}-\frac{1}{2}g_{\mu\nu}+g_{\mu\nu}\nabla^\sigma\nabla_\sigma-\nabla_\mu\nabla_\nu\right]\phi^2
\end{equation}
note that for minimally coupled field ($\xi=0$) the energy-momentum tensor is equivalent to the Noether energy-momentum tensor.

Scalar product
\begin{equation}
	\langle \phi_1(x),\,\phi_2(x)\rangle\equiv i\int g^{0\nu}\left(\phi_1\overset{\text{\tiny$\leftrightarrow$}}{\nabla}_\nu\phi_2^*\right)\sqrt{-g}\,\td^3\mathbf{x}
\end{equation}

Let $v(x)$ be a solution of the Klein-Gordon equation, then $v^*(x)$ will also be an (linearly independent) solution. Let $i$ represent the set of parameters that univocally describe a par of solutions $v_i(x),\,v^*(x)$, therefore, the general solution of the Klein-Gordon equation will be of the form
\begin{equation}
	\phi(x)=\sum_ i\left[a_iv_i(x)+a^*_iv^*_i(x)\right]
\end{equation} 
where $a_i,\,a^*_i$ are constant factors that can be written as
\begin{equation}
	a_i = \langle v_i(x),\,\phi(x)\rangle\hspace{1.0cm} a_i^* = \langle v_i^*(x),\,\phi(x)\rangle
\end{equation}
Quantization of the field is done by promoting the fields to operators
\begin{equation}
	\phi(x)\longrightarrow\hat{\phi}(x)\hspace{1.0cm}\Pi(x)\longrightarrow\hat{\Pi}(x)
\end{equation}
this is done by promoting the constant factors to operators as well, that is
\begin{equation}
	a_i\longrightarrow\hat{a}_i\hspace{1.0cm}a_i^*\longrightarrow\hat{a}_i^\dagger
\end{equation}
and therefore
\begin{equation}
	\hat{\phi}(x)=\sum_i\left[\hat{a}_iv_i(x)+\hat{a}_i^\dagger v^*_i(x)\right]
\end{equation}
after the promotion of the fields to operators, commutation relations are imposed; the easiest choice would be to assume canonical quantization relations,
\begin{equation}
	\left[\hat{\phi}(\mathbf{x}),\,\hat{\Pi}(\mathbf{y})\right]=i\hbar \,\delta^3\left(\mathbf{x}-\mathbf{y}\right)\hspace{1.0cm}\left[\hat{\phi}(\mathbf{x}),\,\hat{\phi}(\mathbf{y})\right]=\left[\hat{\Pi}(\mathbf{x}),\,\hat{\Pi}(\mathbf{y})\right]=0
\end{equation}
It would be desirable to obtain a formulation similar to the well known scalar field in a flat background, where the Fock space is generated from a vacuum state and a set of creation and annihilation operators that follow some commutation rules. To do so, we will force the $\hat{a}_i,\,\hat{a}^\dagger_i$ operators to assume this roll, in such a way that
\begin{equation}
	\left[\hat{a}_i,\,\hat{a}_j^\dagger\right]\propto \delta_{ij}\hspace{1.0cm}\left[\hat{a}_i,\,\hat{a}_j\right]=\left[\hat{a}_i^\dagger,\,\hat{a}_j^\dagger\right]=0
\end{equation}
Thanks to the relation between the constant factors $a_i$ and the scalar product $\langle v_i,\,\phi\rangle$, one can obtain
\begin{multline}
	\left[\hat{a}_i,\,\hat{a}_j^\dagger\right]=-\int\left[\left(v_i\hat{\Pi}-g^{0\nu}\left(\nabla_\nu v_i\right)\hat{\phi}\sqrt{-g}\right)\Big|_\mathbf{x},\,\left(v_j^*\hat{\Pi}-g^{0\nu}\left(\nabla_\nu v_j\right)\hat{\phi}\sqrt{-g}\right)\Big|_\mathbf{y}\right]\td^3\mathbf{x}\td^3\mathbf{y}=\\
	=i\hbar\int g^{0\nu}\left(v_i\overset{\text{\tiny$\leftrightarrow$}}{\nabla}_\nu v_j^*\right)\sqrt{-g}\,\td^3\mathbf{x}=\hbar\langle v_i,\,v_j\rangle
\end{multline}
where the field commutators where used. Equivalently 
\begin{equation}
	\left[\hat{a}_i,\,\hat{a}_j\right]=-\hbar\,\langle v_i,\,v_j^*\rangle\hspace{1.0cm}\left[\hat{a}_i^\dagger,\,\hat{a}_j^\dagger\right]=-\hbar\,\langle v_i^*,\,v_j\rangle
\end{equation}

Therefore we must find a set of solutions $\left\{v_i(x),\,v_i^*(x)\right\}$ such that
\begin{equation}
	\langle v_i,\,v_j\rangle\propto \delta_{ij}\hspace{1.0cm}\langle v_i,\,v_j^*\rangle=\langle v_i^*,\,v_j\rangle=0
\end{equation}
With this, we can define the Fock space the usual way, starting with a vacuum state $|0\rangle$ such that the action of the annihilation operation fulfils
\begin{equation}
	\hat{a}_i\,|0\rangle=0\hspace{1.0cm}\forall i
\end{equation}
where single particle states are formed from the creation operator
\begin{equation}
	|i\rangle\equiv \hat{a}^\dagger_i\,|0\rangle
\end{equation}
and multiparticle states like
\begin{equation}
	|i,\,j,\,\hdots\rangle=\hdots \hat{a}_j^\dagger\,\hat{a}_i^\dagger\,|0\rangle
\end{equation}
Since this is a scalar field, one might assume that the states are symmetric (describing boson particles), and this is easily confirmed, since
\begin{equation}
	|i,\,j\rangle=\hat{a}_j^\dagger\,\hat{a}_i^\dagger\,|0\rangle=\left[\hat{a}_i^\dagger,\,\hat{a}_j^\dagger\right]|0\rangle+\hat{a}_i^\dagger\,\hat{a}_j^\dagger|0\rangle=|j,\,i\rangle
\end{equation}
\section{Bogoliubov transformations}
Consider now a second set $\left\{u_i(x),\,u_i^*(x)\right\}$ of solutions to the Klein-Gordon equation such that they meet the inner product rule; the field would then be written as
\begin{equation}
	\phi(x)=\sum_j\left[b_ju_j(x)+b_j^*u_j^*(x)\right]
\end{equation}
quantization of the field and creation and annihilation is straightforward. The relation between the $v$ and $u$ solutions would be
\begin{equation}
	v_i(x)\equiv\sum_j\left[\alpha_{ij}u_j(x)+\beta_{ij}u^*_j(x)\right]
\end{equation}
where $\alpha_{ij}$ and $\beta_{ij}$ are known as Bogoliubov coefficients, that can be obtained as
\begin{equation}
	\alpha_{ij}\propto\langle v_i,\,u_j\rangle\hspace{1.0cm}\beta_{ij}\propto -\langle v_i,\,u_j^*\rangle
\end{equation}

Since the field is the same independently of the mode set chosen
\begin{equation}
	\sum_i\left[\hat{a}_iv_i(x)+\hat{a}_i^\dagger v^*_i(x)\right]=\sum_j\left[\hat{b}_ju_j(x)+\hat{b}_j^\dagger u_j^*(x)\right]
\end{equation}
and, as a result of the orthogonality of the mode functions
\begin{equation}
	\hat{a}_i=\sum_j\left(\alpha_{ij}^*\hat{b}_j-\beta_{ij}^*\hat{b}_j^\dagger\right)\hspace{1.0cm}\hat{a}_i^\dagger=\sum_j\left(-\beta_{ij}\hat{b}_j+\alpha_{ij}\hat{b}_j^\dagger\right)
\end{equation}
creation and annihilation commutation relations give new restrictions to the Bogoliubov coefficients
\begin{equation}
	\left[\hat{a}_i,\,\hat{a}_j^\dagger\right]\propto\delta_{ij}\implies \sum_k\left(\alpha_{ik}^*\alpha_{jk}-\beta^*_{ik}\beta_{jk}\right)\propto\delta_{ij}
\end{equation}
\begin{equation}
	\left[\hat{a}_i,\,\hat{a}_j\right]=0\implies \sum_k\left(\alpha_{jk}^*\beta_{ik}^*-\alpha_{ik}^*\beta_{jk}^*\right)=0
\end{equation}

Now, the relevance of the Bogoliubov transformations comes from the fact that the vacuum in the $u$ solutions, have (in general) $v$ particles,
\begin{equation}
	\langle_u0|\hat{N}_v|_u0\rangle=\sum_i\langle_u0|\hat{a}_i^\dagger\hat{a}_i|_u0\rangle=\sum_{i}\left[\sum_{jk}\beta_{ij}\beta^*_{ik}\langle_u0|\hat{b}_j\hat{b}_k^\dagger|_u0\rangle\right]\propto \sum_{ij}|\beta_{ij}|^2
\end{equation}
therefore, there is not a unique vacuum.

\section{A leap towards a continuum}
Until now, it has been considered that the set of Klein-Gordon solutions could be categorised by a discrete set of parameters $i$, from a standard course in QFT, one of the main results is the fact that the solutions of the flat Klein-Gordon equations can be parametrised by a continuous $3$-dimensional vector $\mathbf{k}$ (which is interpreted to be the momentum of the particle). Since all computations in this section where made by considering a discrete set of parameters, it is relevant to consider the continuum case.

A common computation in many fields of physics is the determination of the density of states $D(\mathbf{k})$ describing the number of modes with momentum between $\mathbf{k}$ and $\mathbf{k}+\td \mathbf{k}$. Consider a system with volume $V$, where the field goes to zero at its boundary; in this case, the permitted values of momenta must meet
\begin{equation}
	k^i=n^i\frac{\pi\hbar}{V^{\sfrac{1}{3}}}\,,\hspace{1.0cm}n^i\in\mathbb{Z}
\end{equation}
Let $N(k)$ be the number of states with momentum modulus less than $k$, that is, the states such that
\begin{equation}
	n=\sqrt{\left(n^1\right)^2+\left(n^2\right)^2+\left(n^3\right)^2}<k\frac{V^{\sfrac{1}{3}}}{\pi\hbar}
\end{equation}
considering a flat momentum space\footnote{In contrast to modified theories of relativity in which this is not the case, like the $\kappa$-Poincaré relativity.} and a large enough volume, $N(k)$ will be essentially equal to an eight of the volume of a sphere with radius $kV^{\sfrac{1}{3}}\sfrac{}{\pi\hbar}$, that is
\begin{equation}
	N(k)\approx\frac{1}{8}\frac{4}{3}\pi\left(k\frac{V^{\sfrac{1}{3}}}{\pi\hbar}\right)^3=\frac{V}{6\pi^2\hbar^3}k^3
\end{equation} 
meaning, that the density of states will be
\begin{equation}
	D(\mathbf{k})\equiv D(k)=\frac{\td N(k)}{\td k}\approx\frac{V}{2\pi^2\hbar^3}k^2
\end{equation}

With this, one could approximate a discrete sum over a parameter $i$ to an integral over a continuum $\mathbf{k}$
\begin{equation}
	\sum_i f_i=\int_0^\infty D(k)f_k\td k\approx\int_0^\infty\frac{V}{2\pi^2\hbar^3} f_k k^2\td k\equiv\int\frac{\td^3\mathbf{k}}{(2\pi\hbar)^3}f_\mathbf{k}
\end{equation}
where it has been defined.
\begin{equation}
	4\pi V f_k k^2\equiv \int_{\theta=0}^{2\pi}\int_{\varphi=0}^\pi f_\mathbf{k}\sin\varphi \td\theta\td\varphi
\end{equation}

therefore $\sfrac{\td^3\mathbf{k}}{(2\pi\hbar)^3}$ is to be understood as the volume element of the momentum space.