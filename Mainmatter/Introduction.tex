\section{Matter-Gravity Action}
Consider a dynamic universe consisting of dark energy characterized by a cosmological constant $\Lambda$ and some material content described by a Lagrangian density $\mathcal{L}_{\text{M}}$. The action associated with such system would be
\begin{equation}\label{eq: General Action}
	S=\int\left[\frac{1}{2\kappa}\left(R-2\Lambda\right)+\mathcal{L}_{\text{M}}\right]\sqrt{-g}\,\td^4 x,
\end{equation}
where $\kappa\equiv\frac{8\pi G}{c^4}$ is known as the Einstein gravitational constant.

The equations that would describe the classical dynamics of the system can be obtained by variations of the action presented in \cref{eq: General Action} and the stationary-action principle, which states that the path taken by the system will result in $\delta S=0$. The equations of motion of the matter fields are the Euler-Lagrange equations, obtained from a Lagrangian density of matter be described by some set $\left\{\phi^\alpha(x)\right\}$ which depends on the fields and their covariant derivatives, i.e.
\begin{equation}
	\mathcal{L}_{\text{M}}=\mathcal{L}_{\text{M}}\big[\phi^\alpha(x),\,\nabla_\mu\phi^\alpha(x)\big],
\end{equation}
which can be derived from the variations of the action $S$ with respect of $\phi^\alpha$
\begin{equation}
	\delta S=\int\left[\frac{\partial \mathcal{L}_\text{M}}{\partial \phi^\alpha}\delta\phi^\alpha+\frac{\partial \mathcal{L}_\text{M}}{\partial\left(\nabla_\mu\phi^\alpha\right)}\nabla_\mu\left(\delta\phi^\alpha\right)\right]\sqrt{-g}\,\td^4x.
\end{equation}
After applying the generalized Gauss Theorem, the stationary-action principle leads to the aforementioned Euler-Lagrange equations
\begin{equation}\label{eq: Euler-Lagrange}
	\frac{\partial\mathcal{L}_{\text{M}}}{\partial\phi^\alpha}-\nabla_\mu\left[\frac{\partial \mathcal{L}_{\text{M}}}{\partial\left(\nabla_\mu\phi^\alpha\right)}\right]=0.
\end{equation}

 
On the other hand, variations of $S$ with respect to the inverse metric ($g^{\mu\nu}$) results in
\begin{multline}
	\delta S=\int \left[\frac{1}{2\kappa}\frac{\delta R}{\delta g^{\mu\nu}}+\frac{R}{2\kappa}\frac{1}{\sqrt{-g}}\frac{\delta \sqrt{-g}}{\delta g^{\mu\nu}}-\frac{\Lambda}{\kappa}\frac{1}{\sqrt{-g}}\frac{\delta\sqrt{-g}}{\delta g^{\mu\nu}}+\frac{\delta\mathcal{L}_{\text{M}}}{\delta g^{\mu\nu}}+\frac{\mathcal{L}_{\text{M}}}{\sqrt{-g}}\frac{\delta\sqrt{-g}}{\delta g^{\mu\nu}}\right]\delta g^{\mu\nu}\,\sqrt{-g}\,\td^4x;
\end{multline}
and again, by imposing $\delta S=0$ and considering that (up to pure derivative terms)
\begin{subequations}
	\begin{gather}
		\frac{\delta R}{\delta g^{\mu\nu}}=R_{\mu\nu},\hspace{1.0cm}\frac{1}{\sqrt{-g}}\frac{\delta\sqrt{-g}}{\delta g^{\mu\nu}}=-\frac{1}{2}g_{\mu\nu},\tag{\theequation \,\,a,b}
	\end{gather}
\end{subequations}
one obtains the Einstein field equations
\begin{equation}\label{eq: Einstein Field Equations}
	R_{\mu\nu}-\frac{1}{2}g_{\mu\nu}R+\Lambda g_{\mu\nu}=-2\frac{8\pi G}{c^4}\left(\frac{\delta\mathcal{L}_{\text{M}}}{\delta g^{\mu\nu}}-\frac{1}{2}\mathcal{L}_{\text{M}}g_{\mu\nu}\right)
\end{equation}
which are most commonly written in terms of the Hilbert energy-momentum tensor
\begin{equation}\label{eq: Hilbert energy-momentum tensor}
	T_{\mu\nu}\equiv \mathcal{L}_{\text{M}}g_{\mu\nu}-\frac{\delta\mathcal{L}_{\text{M}}}{\delta g^{\mu\nu}}=\frac{-2}{\sqrt{-g}}\frac{\delta \left(\mathcal{L}_{\text{M}}\sqrt{-g}\right)}{\delta g^{\mu\nu}}.
\end{equation}
This tensor is the source of the spacetime curvature, and must not be confused with the Noether's energy-momentum tensor since the two are not, in general, equivalent \cite{Energy-MomentumTensor}, but upon integration of the corresponding conserved currents, results are the same \cite{NoetherVsHilbert}. In addition of being symmetric, the Hilbert energy-momentum tensor is covariantly conserved, i.e.
\begin{equation}
	\nabla_\mu T^{\mu\nu}=0;
\end{equation}
this fact is of great use once the material Hamiltonian $\mathcal{H}_{\text{M}}$ is defined:
\begin{equation}\label{eq: General Hamiltonian}
	\mathcal{H}_{\text{M}}\equiv \int T^{00}c\,\sqrt{-g}\,\td^3\mathbf{x},
\end{equation}
since it will later be used to spawn the Fock space after the quantization procedure; and thus assure no energy loses will be present on the theory.
\section{Construction of Covariant Actions}
Standard quantum field theory alongside the Standard Model of particle physics is one of (if not the) best tested theories of physics, reason which, one doesn't need to reinvent the actions that are used on it, only a small tweak is needed to make the theory general covariant; it has been previously given for granted that the  volume element will be $\sqrt{-g}\,\td^4x$; but there is another consideration, the derivatives cannot be simply $\partial_\mu$ since that is not (in general covariant). To correctly define a covariant derivative $\nabla_\mu$, one must introduce two elements: the first one is the known Christoffel symbols $\Gamma_{\alpha\beta}^\sigma$ which will contract the tensorial nature of the field; and the second one is the spin connection, given by
\begin{equation}
	\Gamma_\nu\equiv \frac{1}{2}\Sigma^{AB}\omega_{AB\mu};
\end{equation}
where $\Sigma^{AB}$ are to be understood as the Lorentz generators (the uppercase Latin indexes represent sums over a Minkowski background), and $\omega_{AB\mu}$ is the so called torsion free spin connection, defined as
\begin{equation}
	\omega_{AB\mu}\equiv e^\nu_A\left(\partial_\mu e_{B\nu}-\Gamma^\sigma_{\nu\mu}e_{B\sigma}\right).
\end{equation}
The new vector fields $e^\mu_A$ are known as the tetrad formalism coefficients, defined to transform general tensors to a local flat manifold, i.e.
\begin{equation}
	g_{\mu\nu}=e^A_\mu e^B_\nu \eta_{ab}.
\end{equation}
As a last note, if the field is coupled to a vector field $A_\mu$ the covariant derivative must be redefined as $\nabla_\mu'\equiv\nabla_\mu-\frac{i}{\hbar}eA_\mu$, where $e$ would be the coupling constant.

\begin{comment}
	
	Standard quantum field theory alongside the Standard Model of particle physics is one of (if not the) best tested theories of physics, reason which, one doesn't need to reinvent the actions that are used on it, only a small tweak is needed to make the theory general covariant; it has been previously given for granted that the  volume element will be $\sqrt{-g}\,\td^4x$, but there is another consideration, the derivatives, since in a general spacetime, they are not covariant, and thus, the inclusion of a covariant derivative $\nabla_\mu$ is needed.
	
	Let $\Xi(x)$ be a tensorial field such that under a local Lorentzian transformation $\Lambda(x)$ transforms as $\tilde{\Xi}\equiv\rho_\Lambda\Xi$. The usual covariant derivative of the transformed field will be
	\begin{multline}\label{eq: covariant derivative general}
		\nabla_\mu\tilde\Xi^{a_1,a_2,\hdots}_{\quad b_1,b_2,\hdots}\equiv \partial_\mu\tilde{\Xi}^{a_1,a_2,\hdots}_{\quad b_1,b_2,\hdots}+\Gamma^{a_1}_{\sigma\mu}\tilde\Xi^{\sigma,a_2,\hdots}_{\quad b_1,b_2,\hdots}+\hdots-\Gamma^{\sigma}_{b_1\mu}\tilde\Xi^{a_1,a_2,\hdots}_{\quad \sigma,b_2,\hdots}-\hdots=\\
		=\left(\partial_\mu\rho_\Lambda\right)\Xi^{a_1,a_2,\hdots}_{\quad b_1,b_2,\hdots}+\rho_\Lambda\partial_\mu\Xi^{a_1,a_2,\hdots}_{\quad b_1,b_2,\hdots}+\Gamma^{a_1}_{\sigma\mu}\tilde\Xi^{\sigma,a_2,\hdots}_{\quad b_1,b_2,\hdots}+\hdots-\Gamma^{\sigma}_{b_1\mu}\tilde\Xi^{a_1,a_2,\hdots}_{\quad \sigma,b_2,\hdots}-\hdots.
	\end{multline}
	From this one can conclude that, for $\nabla_\mu\tilde{\Xi}(x)$ to transform as a tensor, one should redefine the covariant derivative $\nabla_\mu$ to add a term $\Gamma_\mu$ such that it transforms as
	\begin{equation}\label{eq: connection transformation rule}
		\tilde{\Gamma}_\mu\equiv \rho_\Lambda\Gamma_\mu\rho_\Lambda^{-1}-\left(\partial_\mu\rho_\Lambda\right)\rho_\Lambda^{-1},
	\end{equation}
	such term (that we should call connection), can be written as
	\begin{equation}\label{eq: connection}
		\Gamma_\mu\equiv \frac{1}{2}\Sigma^{AB}\omega_{AB\mu};
	\end{equation}
	where $\Sigma^{AB}$ are to be understood as the Lorentz generators (the uppercase Latin indexes represent sums over a Minkowski background), and $\omega_{AB\mu}$ is the so called torsion free spin connection, defined as
	\begin{equation}
		\omega_{AB\mu}\equiv e^\nu_A\left(\partial_\mu e_{B\nu}-\Gamma^\sigma_{\nu\mu}e_{B\sigma}\right).
	\end{equation}
	The new vector fields $e^\mu_A$ are known as the tetrad formalism coefficients, defined to transform general tensors to a local flat manifold, i.e.
	\begin{equation}
		g_{\mu\nu}=e^A_\mu e^B_\nu \eta_{ab}.
	\end{equation}
	
	Therefore, we define the covariant derivative as
	\begin{equation}\label{eq: Covariant derivative}
		\nabla_\mu\Xi^{a_1,a_2,\hdots}_{\quad b_1,b_2,\hdots}\equiv \partial_\mu \Xi^{a_1,a_2,\hdots}_{\quad b_1,b_2,\hdots}+\Gamma_\mu+\Gamma^{a_1}_{\sigma\mu}\tilde\Xi^{\sigma,a_2,\hdots}_{\quad b_1,b_2,\hdots}+\hdots-\Gamma^{\sigma}_{b_1\mu}\tilde\Xi^{a_1,a_2,\hdots}_{\quad \sigma,b_2,\hdots}-\hdots.
	\end{equation}
	
	If the field $\Xi^{a_1,a_2,\hdots}_{\quad b_1,b_2,\hdots}$ is coupled to a vector field $A_\mu$ the covariant derivative must be redefined as $\nabla_\mu'\equiv\nabla_\mu-\frac{i}{\hbar}eA_\mu$, where $e$ would be the coupling constant.
\end{comment}


\subsection{Some Basic Examples}
\subsubsection{Scalar Field}
The very first example given for a classical field is usually a (real) free scalar field $\phi(x)$ with some mass $m$; whose dynamics are given by the following action
\begin{equation}\label{eq: General scalar field action}
	S\big[\phi\big]=\int\frac{1}{2}\Big[\partial_\nu\phi\,\partial^\nu\phi-\mu^2\phi^2-\xi R\phi^2\Big]\sqrt{-g}\;\td^4 x.
\end{equation}
The construction of such action arises from its Minkowskian counterpart (a primary study of the standard quantum field theory can be found in the appendix); since the field in question is scalar, the covariant derivative \ref{eq: Covariant derivative} its simply $\partial_\nu$, the massive term of the action is dependant on a parameter $\mu\equiv\sfrac{mc}{\hbar}$, and a term is added as a coupling to gravity (through the Ricci scalar $R$) with a coupling constant $\xi$ \footnote{The field is said to be minimally coupled to gravity if $\xi=0$ and nonminimally coupled otherwise.}.

The inclusion of such coupling is not a mere curiosity, since it's been proven \cite{Bunch_Renorm} that a self interactive $\lambda\phi^4$ theory needs a term proportional to $R\phi^2$ to be renormalizable. Besides this, the addition of a term proportional to $R\phi^2$ adds a new symmetry to de action, since for a massless field $\mu=0$ with a coupling constant $\xi=\sfrac{1}{6}$, the action is invariant under conformal transformations, i.e.
\begin{equation}\label{eq: Conformal Transformation}
	g_{\mu\nu}\rightarrow \tilde{g}_{\mu\nu}\equiv \Omega^2(x)g_{\mu\nu},
\end{equation}
to prove this, lets first obtain the equations of motion using the Euler-Lagrange equation \ref{eq: Euler-Lagrange}, resulting in the generalized Klein-Gordon equation
\begin{equation}\label{eq: Klein-Gordon General}
	\big[\partial_\nu\partial^\nu-\mu^2-\xi R\big]\phi=0,
\end{equation}
and then, a conformal transformation can be made to then, considering that the field will transform as $\phi\rightarrow\tilde{\phi}=\Omega^\beta\phi$, resulting on the following expression:
\begin{multline}
	0=\mu^2\Omega^{\beta-2}\left(\Omega^2-1\right)\phi+2\left(1+\beta\right)\Omega^{\beta-3}\partial^\nu\Omega\partial_\nu\phi+\\
	+(6\xi+\beta)\Omega^{\beta-3}\left(\partial_\nu\partial^\nu\Omega\right)\phi+\beta(1+\beta)\Omega^{\beta-4}\partial_\nu\Omega\partial^\nu\Omega\,\phi.
\end{multline}
Considering a massless field, a solution of this equation corresponds to the following values:
\begin{subequations}
	\begin{gather}
		\beta=-1,\hspace{1.0cm} \xi=\frac{1}{6},\tag{\theequation \,\,a,b}
	\end{gather}
\end{subequations}
proving the conformal invariance for such scenario.

From its definition in equation \ref{eq: Hilbert energy-momentum tensor} and the action \ref{eq: General scalar field action}, one can obtain the expression for the energy momentum tensor\footnote{Note that in a Minkowski background, the term $R\phi^2$ present in the action written in the equation \ref{eq: General scalar field action}, vanishes; but the energy momentum tensor $T_{\mu\nu}$ differs from the standard expression by a pure derivative term. The new tensor is known as the improved energy-momentum tensor.}
\begin{equation}\label{eq: Energy-Momentum scalar}
	T_{\mu\nu}=\partial_\mu\phi\,\partial_\nu\phi -\frac{1}{2}g_{\mu\nu}\left[\partial^\sigma\phi\partial_\sigma\phi-\mu^2\phi^2\right]+\xi\left[-R_{\mu\nu}+\frac{1}{2}g_{\mu\nu}R-g_{\mu\nu}\partial^\sigma\partial_\sigma+\partial_\mu\partial_\nu\right]\phi^2,
\end{equation}
which has an interesting property of its trace,
\begin{equation}
	T^\nu_\nu=\frac{1}{2}\left(6\xi-1\right)\partial_\sigma\partial^\sigma\phi^2+\mu^2\phi^2,
\end{equation}
as it is zero, for a conformal theory.
\subsubsection{Dirac Field}
For spin $\sfrac{1}{2}$ particles, the Lorentz generators are
\begin{equation}
	\Sigma^{AB}=-\frac{i}{2}\sigma^{AB}=\frac{1}{4}\left[\gamma^A,\,\gamma^B\right],
\end{equation}
where $\Gamma^A$ are the flat gamma matrices. Therefore the covariant derivative \ref{eq: Covariant derivative} and the connection \ref{eq: connection} can be written as
\begin{subequations}
	\begin{gather}
		\nabla_\mu\equiv\partial_\mu+\Gamma_\mu,\quad \Gamma_\mu=\frac{1}{8}\omega_{AB\mu}\left[\gamma^A,\,\gamma^B\right].
	\end{gather}
\end{subequations}

Taking the Dirac theory as inspiration, one could define de Dirac action in curved spacetimes as
\begin{equation}
	S[\psi]=\int \bar{\psi}\left[i\gamma^\nu\nabla_\nu-\mu\right]\psi\sqrt{-g}\td^4x,
\end{equation}
where $\Gamma^\nu\equiv \gamma^Ae^\nu_A$ are the general gamma functions, which follow the next relation similar to their flat counterparts,
\begin{equation}
	\left\{\gamma^\mu,\,\gamma^\nu\right\}=2g^{\mu\nu}.
\end{equation}

From the Euler-Lagrange equation \ref{eq: Euler-Lagrange}, its obtained the generalized Dirac equation
\begin{equation}
	\left[i\gamma^\mu\left(\partial_\mu+\Gamma_\mu\right)-\mu\right]\psi=0.
\end{equation}
And from its definition in equation \ref{eq: Hilbert energy-momentum tensor}, the energy momentum tensor will have \cite[sec.\,3.8]{BirrelDavies} the following expression
\begin{equation}
	T_{\mu\nu}=\frac{1}{4}i\left\{\bar{\psi}\left(\gamma_\mu\nabla_\nu-\gamma_\nu\nabla_\mu\right)-\left[\left(\nabla_\mu\bar{\psi}\right)\gamma_\nu-\left(\nabla_\nu\bar{\psi}\right)\gamma_\mu\right]\right\}\psi,
\end{equation}
with a trace of the form $T^\mu_\mu=4m\bar{\psi}\psi$, which is traceless for a massless field.

A particularly interesting outcome of this field is the so called Schrödinger-Dirac equation, result of squaring the generalized Dirac operator $\left[i\gamma^\mu\left(\partial_\mu+\Gamma_\mu\right)-\mu\right]$ just as its done in a Minkowskian background to recover the Klein-Gordon equation,
\begin{equation}
	\left[\nabla_\nu\nabla^\nu-\mu^2-\frac{1}{4}R\right]\psi=0,
\end{equation}
this expression (known as the Weitzenböck formula) gives another "natural" choice for the scalar field coupling to gravity $\xi$; to obtain such value, one can compare it with the generalized Klein-Gordon equation \ref{eq: Generalized Klein-Gordon} finding $\xi=\sfrac{1}{4}$.

\subsubsection{Electromagnetic Field}
Having previously studied the Dirac field, it is expected to also include the electromagnetic field, which is described by the same action as in the Minkowskian background, that is,
\begin{equation}\label{eq: Electromagnetic action}
	S[A_\mu]=\int\left(-\frac{1}{4 c}F_{\mu\nu}F^{\mu\nu}+\mathcal{L}_{\text{Gauge}}\right)\sqrt{-g}\td^4x,
\end{equation}
where $\mathcal{L}_{\text{Gauge}}$ is a gauge fixing term of the form
\begin{equation}
	\mathcal{L}_{\text{Gauge}}=-\frac{1}{2\alpha}\left(\nabla_\nu A^\nu\right)^2.
\end{equation}
The Faraday tensor is defined as
\begin{equation}
	F_{\mu\nu}=\nabla_\mu A_\nu-\nabla_\nu A_\mu=\partial_\mu A_\nu-\partial_\nu A_\mu,
\end{equation}
where the last equality is a result of the symmetry of the lower indices on the Christoffel symbols.


The equations of motion resulting from the action \ref{eq: Electromagnetic action} and Euler-Lagrange \ref{eq: Euler-Lagrange} are
\begin{equation}
	\nabla^\nu\nabla_\nu A_\mu+R_\mu^\sigma A_\sigma-\left(1-\alpha\right)\nabla_\mu\nabla^\nu A_\nu=0.
\end{equation}

And finally, as for completeness, the energy-momentum tensor \cite[sec.\,3.8]{BirrelDavies} given by \ref{eq: Hilbert energy-momentum tensor} is
\begin{multline}
	T_{\mu\nu}=-\left(F_{\mu\alpha}F^{\alpha\nu}-\frac{1}{4}g_{\mu\nu}F_{\alpha\beta}F^{\alpha\beta}\right)+\\
	+\alpha\left\{A_\mu\left(\nabla_\nu\nabla_\rho A^\rho\right)+\left(\nabla_\mu\nabla_\rho A^\rho\right)A_\nu-g_{\mu\nu}\left[A^\rho\left(\nabla\rho\nabla_\sigma A^\sigma\right)+\frac{1}{2}\left(\nabla_\rho A^\rho\right)^2\right]\right\};
\end{multline}
with a trace $T^\mu_\mu=-2\alpha\nabla_\nu\left(A^\nu \nabla_\rho A^\rho\right)$, which is contributed by the gauge fixing term only.
\section{Scalar field Quantization}
Thanks to its simplicity, the scalar field is a great field to work with, with the intent of showing some properties of a theory. For that reason, for what follows, all work will be done considering a real scalar field described by the action \ref{eq: General scalar field action}.

Now, let $v(x)$ be a solution of the generalized Klein-Gordon equation \ref{eq: Generalized Klein-Gordon}, then its complex conjugated $\bar{v}$ will also be an independent solution. Now consider $i$ to be a set of parameters that univocally describe a par of solutions $v_i,\,\bar v_i$ in such a way that the most general solution of \ref{eq: Generalized Klein-Gordon} will be
\begin{equation}
	\phi(x)=\sum_ i\left[a_iv_i(x)+\bar a_i\bar v_i(x)\right],
\end{equation}
where $a_i$ and $\bar a_i$ are constant factors, determined by the following external binary operation
\begin{equation}\label{eq: General Scalar Inner Product}
	\langle \phi_1(x),\,\phi_2(x)\rangle\equiv \frac{i}{\hbar}\int g^{0\nu}\left(\phi_1\overset{\text{\tiny$\leftrightarrow$}}{\partial}_\nu\bar \phi_2\right)\sqrt{-g}\,\td^3\mathbf{x},
\end{equation}
such that
\begin{subequations}
	\begin{gather}
		a_i = \langle v_i(x),\,\phi(x)\rangle,\quad \bar a_i = \langle \bar v_i(x),\,\phi(x)\rangle.\tag{\theequation \,\,a,b}
	\end{gather}
\end{subequations}

The quantization procedure is done by promoting the field $\rchi$ and its conjugate momentum $\Pi\equiv\partial_{ct}\rchi$ to operators
\begin{equation}
	\phi(x)\longrightarrow\hat{\phi}(x),\hspace{1.0cm}\Pi(x)\longrightarrow\hat{\Pi}(x),
\end{equation}
by promoting the constant factors to operators as well, that is
\begin{equation}
	a_i\longrightarrow\hat{a}_i,\hspace{1.0cm}\bar a_i\longrightarrow\hat{a}_i^\dagger,
\end{equation}
and therefore
\begin{equation}
	\hat{\phi}(x)=\sum_i\left[\hat{a}_iv_i(x)+\hat{a}_i^\dagger \bar v_i(x)\right].
\end{equation}
One the promotion of the field to operators have been done, commutation relations between those operators must be imposed; the easiest choice would be to assume canonical quantization relations, that is,
\begin{subequations}\label{eq: General canonical commutator}
	\begin{gather}
		\left[\hat{\phi}(\mathbf{x}),\,\hat{\Pi}(\mathbf{y})\right]=i\hbar \,\delta^3\left(\mathbf{x}-\mathbf{y}\right)\hspace{1.0cm}\left[\hat{\phi}(\mathbf{x}),\,\hat{\phi}(\mathbf{y})\right]=\left[\hat{\Pi}(\mathbf{x}),\,\hat{\Pi}(\mathbf{y})\right]=0.\tag{\theequation \,\,a-c}
	\end{gather}
\end{subequations}
It would be desirable to obtain a formulation similar to the well known scalar field in a Minkoswskian background, where the Fock space is generated from a vacuum state and a set of creation and annihilation operators that follow some commutation rules. To do so, we will force the $\hat{a}_i,\,\hat{a}^\dagger_i$ operators to assume this roll, in such a way that
\begin{subequations}\label{eq: general a operator commutator rules}
	\begin{gather}
		\left[\hat{a}_i,\,\hat{a}_j^\dagger\right]\propto \delta_{ij},\hspace{1.0cm}\left[\hat{a}_i,\,\hat{a}_j\right]=\left[\hat{a}_i^\dagger,\,\hat{a}_j^\dagger\right]=0.\tag{\theequation \,\,a-c}
	\end{gather}
\end{subequations}
Thanks to the relation between the constant factors $a_i$ and the operation $\langle v_i,\,\phi\rangle$, one can obtain the following relation
\begin{multline}
	\left[\hat{a}_i,\,\hat{a}_j^\dagger\right]=-\frac{1}{\hbar^2}\int\left[\left(v_i\hat{\Pi}-g^{0\nu}\left(\partial_\nu v_i\right)\hat{\phi}\sqrt{-g}\right)\Big|_\mathbf{x},\,\left(\bar v_j\hat{\Pi}-g^{0\nu}\left(\partial_\nu v_j\right)\hat{\phi}\sqrt{-g}\right)\Big|_\mathbf{y}\right]\td^3\mathbf{x}\td^3\mathbf{y}=\\
	=\frac{i}{\hbar}\int g^{0\nu}\left(v_i\overset{\text{\tiny$\leftrightarrow$}}{\partial}_\nu \bar v_j\right)\sqrt{-g}\,\td^3\mathbf{x}=\langle v_i,\,v_j\rangle,
\end{multline}
where the field commutators where used. Equivalently
\begin{subequations}
	\begin{gather}
		\left[\hat{a}_i,\,\hat{a}_j\right]=-\langle v_i,\,\bar v_j\rangle,\quad \left[\hat{a}_i^\dagger,\,\hat{a}_j^\dagger\right]=-\langle \bar v_i,\,v_j\rangle. \tag{\theequation \,\,a,b}
	\end{gather}
\end{subequations}

Therefore we must find a set of solutions $\left\{v_i(x),\,\bar v_i(x)\right\}$ such that
\begin{subequations}\label{eq: Operational Rules}
	\begin{gather}
		\langle v_i,\,v_j\rangle\propto \delta_{ij},\quad \langle v_i,\,\bar v_j\rangle=\langle \bar v_i,\,v_j\rangle=0. \tag{\theequation \,\,a-c}
	\end{gather}
\end{subequations}
With this, we can define the Fock space the usual way, starting with a vacuum state $|0\rangle$ such that the action of the annihilation operation fulfils
\begin{equation}
	\hat{a}_i\,|0\rangle=0\hspace{1.0cm}\forall i
\end{equation}
where single particle states are formed from the creation operator
\begin{equation}
	|i\rangle\equiv \hat{a}^\dagger_i\,|0\rangle
\end{equation}
and multiparticle states like
\begin{equation}
	|i,\,j,\,\hdots\rangle=\hdots \hat{a}_j^\dagger\,\hat{a}_i^\dagger\,|0\rangle
\end{equation}
Since this is a scalar field, one might assume that the states are symmetric (describing boson particles), and this is easily confirmed, since
\begin{equation}
	|i,\,j\rangle=\hat{a}_j^\dagger\,\hat{a}_i^\dagger\,|0\rangle=\left[\hat{a}_i^\dagger,\,\hat{a}_j^\dagger\right]|0\rangle+\hat{a}_i^\dagger\,\hat{a}_j^\dagger|0\rangle=|j,\,i\rangle
\end{equation}
\subsection{Bogoliubov Transformations}
Consider now a second set $\left\{u_i(x),\,\bar u_i(x)\right\}$ of solutions to the Klein-Gordon equation \ref{eq: Generalized Klein-Gordon} such that they meet the operational rules \ref{eq: Operational Rules}; the field would then be written as
\begin{equation}
	\phi(x)=\sum_j\left[b_ju_j(x)+\bar b_j\bar u_j(x)\right].
\end{equation}
Quantization of the field is straightforward and equivalent to the method previously presented. The relation between the $v$ and $u$ solutions (mode functions) will be
\begin{equation}
	v_i(x)\equiv\sum_j\left[\alpha_{ij}u_j(x)+\beta_{ij}\bar u_j(x)\right],
\end{equation}
where $\alpha_{ij}$ and $\beta_{ij}$ are known as Bogoliubov coefficients, that can be obtained as
\begin{subequations}\label{eq: Bogoliubov Coefficients General}
	\begin{gather}
		\alpha_{ij}\propto\langle v_i,\,u_j\rangle\hspace{1.0cm}\beta_{ij}\propto -\langle v_i,\,\bar u_j\rangle.\tag{\theequation \,\,a,b}
	\end{gather}
\end{subequations}

Since the field is the same independently of the mode set chosen:
\begin{equation}
	\sum_i\left[\hat{a}_iv_i(x)+\hat{a}_i^\dagger v^*_i(x)\right]=\sum_j\left[\hat{b}_ju_j(x)+\hat{b}_j^\dagger u_j^*(x)\right]
\end{equation}
and, as a result of the orthogonality of the mode functions, the relation between the creation and annihilation operators will be 
\begin{subequations}
	\begin{gather}
		\hat{a}_i=\sum_j\left(\bar \alpha_{ij}\hat{b}_j-\bar \beta_{ij}\hat{b}_j^\dagger\right),\quad \hat{a}_i^\dagger=\sum_j\left(-\beta_{ij}\hat{b}_j+\alpha_{ij}\hat{b}_j^\dagger\right).\tag{\theequation \,\,a,b}
	\end{gather}
\end{subequations}
Applying commutator rules present in equation \ref{eq: general a operator commutator rules}, give new restrictions to the Bogoliubov coefficients
\begin{equation}
	\left[\hat{a}_i,\,\hat{a}_j^\dagger\right]\propto\delta_{ij}\implies \sum_k\left(\bar \alpha_{ik}\alpha_{jk}-\bar \beta_{ik}\beta_{jk}\right)\propto\delta_{ij},
\end{equation}
\begin{equation}
	\left[\hat{a}_i,\,\hat{a}_j\right]=0\implies \sum_k\left(\bar \alpha_{jk}\bar \beta_{ik}-\bar \alpha_{ik}\bar \beta_{jk}\right)=0.
\end{equation}

One might ask what the relevance of this transformation is, and it would be in its right to do so, since it is not a mere mathematical result. To see the reason of this transformation, one could compute the number of $v$ particles that are present in the $u$ vacuum; the computation is given by
\begin{equation}\label{eq: Bogoliubov Number particles general}
	\langle_u0|\hat{N}_v|_u0\rangle=\sum_i\langle_u0|\hat{a}_i^\dagger\hat{a}_i|_u0\rangle=\sum_{i}\left[\sum_{jk}\beta_{ij}\bar \beta_{ik}\langle_u0|\hat{b}_j\hat{b}_k^\dagger|_u0\rangle\right]\propto \sum_{ij}|\beta_{ij}|^2.
\end{equation}
The usual expectation value of a term of the form $\langle0|\hat{N}|0\rangle$ is to be zero, and yet, it has been proven that this is not the case (in general) for the current scenario. The interpretation of such result is that the notion of "particle" is dependent on the choice of solutions of the Klein-Gorton equation; and thus, one could define different vacuum states for different situations.

\begin{comment}
	\subsection{A Leap Towards a Continuum}
	Until now, it has been considered that the set of Klein-Gordon solutions could be categorised by a discrete set of parameters $i$, from a standard course in QFT, one of the main results is the fact that the solutions of the flat Klein-Gordon equations can be parametrised by a continuous $3$-dimensional vector $\mathbf{k}$ (which is interpreted to be the momentum of the particle). Since all computations in this section where made by considering a discrete set of parameters, it is relevant to consider the continuum case.
	
	A common computation in many fields of physics is the determination of the density of states $D(\mathbf{k})$ describing the number of modes with momentum between $\mathbf{k}$ and $\mathbf{k}+\td \mathbf{k}$. Consider a system with volume $V$, where the field goes to zero at its boundary; in this case, the permitted values of momenta must meet
	\begin{equation}
		k^i=n^i\frac{\pi\hbar}{V^{\sfrac{1}{3}}}\,,\hspace{1.0cm}n^i\in\mathbb{N}
	\end{equation}
	Let $N(k)$ be the number of states with momentum modulus less than $k$, that is, the states such that
	\begin{equation}
		n=\sqrt{\left(n^1\right)^2+\left(n^2\right)^2+\left(n^3\right)^2}<k\frac{V^{\sfrac{1}{3}}}{\pi\hbar}
	\end{equation}
	considering a flat momentum space\footnote{In contrast to modified theories of relativity in which this is not the case, like the $\kappa$-Poincaré relativity.} and a large enough volume, $N(k)$ will be essentially equal to an eight of the volume of a sphere with radius $kV^{\sfrac{1}{3}}\sfrac{}{\pi\hbar}$, that is
	\begin{equation}
		N(k)\approx\frac{1}{8}\frac{4}{3}\pi\left(k\frac{V^{\sfrac{1}{3}}}{\pi\hbar}\right)^3=\frac{V}{6\pi^2\hbar^3}k^3
	\end{equation} 
	meaning, that the density of states will be
	\begin{equation}
		D(\mathbf{k})\equiv D(k)=\frac{\td N(k)}{\td k}\approx\frac{V}{2\pi^2\hbar^3}k^2
	\end{equation}
	
	With this, one could approximate a discrete sum over a parameter $i$ to an integral over a continuum $\mathbf{k}$
	\begin{equation}
		\sum_i f_i=\int_0^\infty D(k)f_k\td k\approx\int_0^\infty\frac{V}{2\pi^2\hbar^3} f_k k^2\td k\equiv\int\frac{\td^3\mathbf{k}}{(2\pi\hbar)^3}f_\mathbf{k}
	\end{equation}
	where it has been defined.
	\begin{equation}
		4\pi V f_k k^2\equiv \int_{\theta=0}^{2\pi}\int_{\varphi=0}^\pi f_\mathbf{k}\sin\varphi \td\theta\td\varphi
	\end{equation}
	
	therefore $\sfrac{\td^3\mathbf{k}}{(2\pi\hbar)^3}$ is to be understood as the volume element of the momentum space.
\end{comment}