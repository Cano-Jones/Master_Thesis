
The theory of Quantum Fields in Curved Spacetimes might be the best probe into experimental and observational testing on new Quantum Gravity proposals, standing on the fringe of current physics knowledge. Although mostly accepted by the community, it lacks experimental and observational support; the sheer magnitude of cosmological metrics combined with the measurement precision needed to observe quantum phenomenology creates a technological barrier that has yet to be overcome.   Even though no direct evidence has been found, analogue models of the Unruh effect made in condensed matter systems \cite{UnruhExperiment} have measured the expected radiation predicted by Hawking.

\vspace*{0.25cm}
The framework presented in this thesis is considered as the best approach to date to model the interaction of matter and gravitational fields within current physical models. This is what is expected to be recovered at the infrared range of a quantum gravity unification theory; despite the fact that no definitive model of said theory has been found, there are proposals of phenomenology to be found at low energies (see \cite{QGpheno_Report} for a recent comprehensive report).


\vspace*{0.4cm}
This thesis has looked into simple models like scalar fields or backgrounds with plenty of symmetries, alongside the phenomenology that might results from them, such as the creation of new particles, different vacua definitions and thermal baths for non-inertial observers. Further areas of interest of advanced complexity within the field might include the study of Spinor Bundles, scattering processes; gravitational perturbation theory or any of the countless theorems resulting from this framework, such as the Harrison-Zeldovich theorem, which states that conformal theories do not create massive particles. Said results could have substantial relevance for cosmological evolution and astrophysical formations. It is in this regard that physicists hold the Theory of Quantum Fields in Curved Spacetimes, since it might help to deepen our knowledge on unexpected effects that otherwise would have not been considered, such as black hole evaporation, and to add new questions to guide the research to new horizons, such as the problem with information loss as a result of black hole evaporation.

