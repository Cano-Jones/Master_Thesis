\section{Accelerated Observers and Unruh Temperature}
Flat $1+1$ spacetime, 
\begin{equation}
	\alpha^\mu\equiv\frac{\td^2x^\mu}{\td\tau^2}
\end{equation}
\begin{equation}
	\alpha^2=\left(c\frac{\td^2 t}{\td\tau^2}\right)^2-\left(\frac{\td^2x}{\td\tau^2}\right)^2
\end{equation}
\begin{subequations}
	\begin{gather}
		t(\tau)=t_0+t_1\tau\pm\frac{c}{\alpha}\sinh\left(\frac{\alpha\tau}{c}\right),\quad x(\tau)=x_0+x_1\tau\pm\frac{c^2}{\alpha}\cosh\left(\frac{\alpha\tau}{c}\right).\tag{\theequation \,\,a,b}
	\end{gather}
\end{subequations}
$t_0,t_1,x_0,x_1\in\mathbb{R}$ but since $c^2\td\tau^2=c^2\td t^2-\td x^2$ one deduces that $ct_1=x_1=0$, for simplicity we consider some coordinates in which $ct_0=x_0=0$.

\begin{equation}
	x^2-c^2t^2=\frac{c^2}{\alpha^2}
\end{equation}
\begin{wrapfigure}{r}{0.4\textwidth}
	\centering{
		\definecolor{gridcolor}{RGB}{255 216 234} 
\definecolor{gridlabelcolor}{RGB}{255 24 131}

\definecolor{Rightgrid}{RGB}{141, 195, 227} 
\definecolor{RightEx}{RGB}{54, 98, 245}

\definecolor{Leftgrid}{RGB}{222, 198, 247} 
\definecolor{LeftEx}{RGB}{160, 76, 245}
\begin{tikzpicture}[%
	scale=3,%
	GridRight/.style={draw=Rightgrid,thick},%
	SubRight/.style={draw=Rightgrid,thin},%
	GridLeft/.style={draw=Leftgrid,thick},%
	SubLeft/.style={draw=Leftgrid,thin},%
	ExRight/.style={draw=RightEx, thick},%
	ExLeft/.style={draw=LeftEx, thick},%
	tlabels/.style={pos=0.88,above,sloped,yshift=-0.3ex,RightEx},%
	label/.style={%
		postaction={%
			decorate,%
			transform shape,%
			decoration={%
				markings,%
				mark=at position .65 with \node #1;%
			}%
		}%
	},%
	]%
	\pgfmathdeclarefunction{arcosh}{1}{\pgfmathparse{ln(#1+sqrt(#1+1)*sqrt(#1-1))}}
	\pgfmathsetmacro{\Xmax}{1.2}
	\pgfmathsetmacro{\Tmax}{1.2}
	\pgfmathsetmacro{\g}{1}
	\newcommand\mylabelstyle\tiny
	
	% curves t=constant
	\foreach \t in {-3,-2.9375,...,3}{%
		\path[SubRight] (0,0) -- (\Xmax,{\Xmax*tanh(\g*\t)});
	}
	\foreach \t in {-3,-2.75,...,3}{%
		\path[GridRight] (0,0) -- (\Xmax,{\Xmax*tanh(\g*\t)});
	} 
	\foreach \t in {-3,-2.9375,...,3}{%
		\path[SubLeft] (0,0) -- (-\Xmax,{-\Xmax*tanh(\g*\t)});
	}
	\foreach \t in {-3,-2.75,...,3}{%
		\path[GridLeft] (0,0) -- (-\Xmax,{-\Xmax*tanh(\g*\t)});
	}   
	
	% curves x=constant
	\foreach \xx in {0.05,0.1,...,\Xmax}{%
		\path[SubRight]
		plot[domain=-{arcosh(\Xmax/\xx)/\g}:{arcosh(\Xmax/\xx)/\g}]
		({\xx*cosh(\g*\x)},{\xx*sinh(\g*\x)});  
	}
	\foreach \xx in {0.05,0.1,...,\Xmax}{%
		\path[SubLeft]
		plot[domain=-{-arcosh(\Xmax/\xx)/\g}:{-arcosh(\Xmax/\xx)/\g}]
		({-\xx*cosh(\g*\x)},{-\xx*sinh(\g*\x)});  
	}
	\foreach \xx in {0.2,0.4,...,1}{%
		\path[GridRight]
		plot[domain=-{arcosh(\Xmax/\xx)/\g}:{arcosh(\Xmax/\xx)/\g}]
		({\xx*cosh(\g*\x)},{\xx*sinh(\g*\x)});  
	}
	\foreach \xx in {0.2,0.4,...,1}{%
		\path[GridLeft]
		plot[domain=-{-arcosh(\Xmax/\xx)/\g}:{-arcosh(\Xmax/\xx)/\g}]
		({-\xx*cosh(\g*\x)},{-\xx*sinh(\g*\x)});  
	}
	
	\foreach \xx in {0.55}{%
		\path[ExRight]
		plot[domain=-{arcosh(\Xmax/\xx)/\g}:{arcosh(\Xmax/\xx)/\g}]
		({\xx*cosh(\g*\x)},{\xx*sinh(\g*\x)});  
	}
	
	\foreach \xx in {0.55}{%
		\path[ExLeft]
		plot[domain=-{-arcosh(\Xmax/\xx)/\g}:{-arcosh(\Xmax/\xx)/\g}]
		({-\xx*cosh(\g*\x)},{-\xx*sinh(\g*\x)});  
	}
	
	% curve labels
	
	\foreach \xx in {0.75}{%
		\path[label={[above]{\mylabelstyle $\xi=$const}}]
		plot[domain=-{arcosh(\Xmax/\xx)/\g}:{arcosh(\Xmax/\xx)/\g}]
		({\xx*cosh(\g*\x)},{\xx*sinh(\g*\x)});  
	}
	
	\foreach \xx in {0.75}{%
		\path[label={[below]{\mylabelstyle $\tilde \xi=$const}}]
		plot[domain=-{-arcosh(\Xmax/\xx)/\g}:{-arcosh(\Xmax/\xx)/\g}]
		({-\xx*cosh(\g*\x)},{-\xx*sinh(\g*\x)});  
	}
	
	% X-axis, T-axis, and dashed lines t=+/-infty 
	\draw[very thick,-stealth] (-\Xmax,0) -- (\Xmax,0) node[below] {$x$};
	\draw[thick,-stealth] (0,-\Tmax) -- (0,\Tmax) node[left] {$t$};
	\draw[dashed] (0,0) -- (\Xmax,\Tmax) 
	node[pos=0.37,above,sloped,yshift=-.3ex] {\mylabelstyle$\xi=-\infty$}
	node[tlabels,black] {\mylabelstyle$\eta=\infty$};
	\draw[dashed] (0,0) -- (\Xmax,-\Tmax)
	node[tlabels,black] {\mylabelstyle$\eta=-\infty$};
	
	\draw[dashed] (0,0) -- (-\Xmax,-\Tmax) 
	node[tlabels,black] {\mylabelstyle$\tilde\eta=-\infty$};
	\draw[dashed] (0,0) -- (-\Xmax,\Tmax)
	node[pos=0.37,above,sloped,yshift=-.3ex] {\mylabelstyle$\tilde\xi=-\infty$}
	node[tlabels,black] {\mylabelstyle$\tilde\eta=\infty$};
	
\end{tikzpicture}

		\caption{Rindler Coordinates (Left and Right charts).}
		\label{fig: Rindler Coordinates}
	}
\end{wrapfigure}
\begin{itemize}
	\item $x>c|t|$
	\begin{subequations}
		\begin{gather}
			t(\eta,\,\xi)\equiv \frac{c}{\alpha}\sinh\left(\frac{\alpha\eta}{c}\right)e^{\sfrac{\alpha\xi}{c^2}},\\ x(\eta,\,\xi)\equiv \frac{c^2}{\alpha}\cosh\left(\frac{\alpha\eta}{c}\right)e^{\sfrac{\alpha\xi}{c^2}},\tag{\theequation \,\,b}
		\end{gather}
	\end{subequations}
	\item $x<c|t|$
	\begin{subequations}
		\begin{gather}
			t(\tilde\eta,\,\tilde\xi)\equiv- \frac{c}{\alpha}\sinh\left(\frac{\alpha\tilde\eta}{c}\right)e^{\sfrac{\alpha\tilde\xi}{c^2}},\\ x(\tilde\eta,\,\tilde\xi)\equiv -\frac{c^2}{\alpha}\cosh\left(\frac{\alpha\tilde\eta}{c}\right)e^{\sfrac{\alpha\tilde\xi}{c^2}},\tag{\theequation \,\,b}
		\end{gather}
	\end{subequations}
\end{itemize}


$x^2-c^2t^2=\frac{c^2}{\alpha^2}e^{2\sfrac{\alpha\xi}{c^2}}$
\begin{equation}
	c^2\td \tau^2=e^{\sfrac{\alpha\xi}{c^2}}\left[c^2\td\eta^2-\td\xi^2\right]
\end{equation}

Now consider a massless and minimally coupled scalar field described by the action \ref{eq: General scalar field action}, the equation of motion would be \footnote{As it turns out, for $1+1$ the conformal invariance is obtained by a massless minimally coupled theory.}
\begin{equation}
	e^{-2\sfrac{\alpha\xi}{c^2}}\left[\partial^2_{c\eta}-\partial^2_\xi\right]\phi=0
\end{equation}
use of null coordinates $u\equiv c\eta-\xi$ $v\equiv c\eta+\xi$ 
\begin{subequations}
	\begin{gather}
		\phi_\omega^u\equiv e^{i\omega u\,\hbar^{-1}},\quad \phi_\omega^v\equiv  a^v_\omega e^{i\omega v\,\hbar^{-1}},\quad\phi_\omega^{\tilde u}\equiv a^{\tilde u}_\omega e^{i\omega \tilde u\,\hbar^{-1}},\quad\phi_\omega^{\tilde v}\equiv  a^{\tilde v}_\omega e^{i\omega \tilde v\,\hbar^{-1}},\tag{\theequation \,\,a-d}
	\end{gather}
\end{subequations}




$U\equiv ct-x$ and $V\equiv ct+x$
\begin{subequations}
	\begin{gather}
		U(u,\,v)=-\frac{c^2}{\alpha}e^{-\sfrac{\alpha u}{c^2}},\hspace{1.0cm} V(u,\,v)=\frac{c^2}{\alpha}e^{\sfrac{\alpha v}{c^2}},
	\end{gather}
\end{subequations}
$\phi\equiv\phi_u+\phi_v$
\begin{equation}
	\phi_u\equiv\int_0^\infty \frac{\td \omega}{(2\pi\hbar)\sqrt{2\omega}}\left\{\Theta(-U)\left[a^u_\omega\phi_\omega^u+\bar a^u_\omega\bar \phi_\omega^u \right]+\Theta(U)\left[a^{\tilde u}_\omega\phi_\omega^{\tilde u}+\bar a^{\tilde u}_\omega\bar \phi_\omega^{\tilde u} \right]\right\}
\end{equation}
\begin{equation}
	\beta_{\Omega\omega}\propto \langle \Theta(-U)\phi^u_\Omega+\Theta(U)\phi^{\tilde u}_\Omega,\bar\phi_\omega^{U}\rangle=\frac{1}{2\hbar^2}\sqrt{\frac{\Omega}{\omega}}\int_{-\infty}^\infty\exp{\left\{i\left[\omega U(u)+\Omega u\right]\hbar^{-1}\right\}}\td u
\end{equation}
$z\equiv i\left(\sfrac{\omega c^2}{\alpha}\right)\exp{\left(\sfrac{a u}{c^2}\right)}$
\begin{equation}
	\beta_{\Omega\omega}\propto \frac{c^2}{\alpha\hbar^2}\Gamma\left(i\frac{\omega c^2}{\alpha\hbar }\right)\sqrt{\frac{\Omega}{\omega}}\left(\frac{c^2\omega}{\alpha\hbar }\right)^{-i\frac{\omega c^2}{\alpha\hbar}}e^{-\frac{\pi\omega c^2}{2\alpha\hbar}}
\end{equation}
\begin{equation}
	N_\omega\equiv\langle_U 0|\hat{a}_\omega^\dagger\hat{a}_\omega|_U0\rangle=\int_0^\infty\frac{\td\omega}{(2\pi\hbar)\sqrt{2\omega}}|\beta_{\omega\omega}|\propto\frac{1}{e^{\frac{2\pi\omega c}{\alpha\hbar}}-1}\delta(0)
\end{equation}
\begin{equation}
	T_0\equiv \frac{\alpha\hbar}{2\pi ck_B}
\end{equation}
This is not the temperature that would be measured by an accelerated Rindler observer; consider the \textit{conserved} energy $E_0$ measured by an observed in an stationary gravitational field, which relates to the energy measured by another observer by $E_0=\sqrt{g_{00}}\,E$, since the thermodynamical relation for the energy and the temperature comes from the entropy $\sfrac{1}{T_0}=\sfrac{\partial S}{E_0}$, then the proper temperature will be $T=\sqrt{g_{00}}\,T_0$; this is known as Tolman's law, and results in
\begin{equation}
	T_{\text{Unruh}}=\sqrt{g_{00}}\,T_0=\left(\alpha e^{-\sfrac{\alpha}{c^2}\xi}\right)\frac{\hbar}{2\pi ck_B}\equiv \frac{a\hbar}{2\pi ck_B}
\end{equation}
\section{Application to Black Holes: Hawking Radiation}
\begin{equation}
	c^2\td\tau^2=\left(1-\frac{2GM}{c^2r}\right)c^2\td t^2-\left(1-\frac{2GM}{c^2r}\right)^{-1}\td r^2-r^2\left(\td\theta^2+\sin^2\theta\td\varphi^2\right)
\end{equation}
$R_S\equiv\sfrac{2GM}{c^2}$

consider a two dimensional black hole
\begin{equation}
	c^2\td\tau^2=\left(1-\frac{R_S}{r}\right)c^2\td t^2-\left(1-\frac{R_S}{r}\right)^{-1}\td r^2
\end{equation}
tortoise coordinate
\begin{equation}
	\td r^*\equiv \left(1-\frac{R_S}{r}\right)^{-1}\td r
\end{equation}
\begin{equation}
	c^2\td\tau^2=\left[1-\frac{R_S}{r\left(r^*\right)}\right]\left[c^2\td t^2-\td r^*\right]
\end{equation}
\footnote{From a non quantum point of view, one could consider the acceleration experienced by an observer situated at $r=R_S$ without angular momentum, which will be equal to $a=\sfrac{GM}{R_S^2}=\sfrac{c^4}{4GM}$}
\begin{equation}
	T_{\text{Hawking}}=\frac{\hbar c^3}{8\pi GMk_B}
\end{equation}

\begin{equation}
	\td S=\left(\frac{\partial S}{\partial M}\right)\td M+\left(\frac{\partial S}{\partial J}\right)\td J+\left(\frac{\partial S}{\partial Q}\right)\td Q
\end{equation}
$Q=J=0$ $E=Mc^2$
\begin{equation}
	\td S=\frac{c^2}{T_{\text{Hawking}}}\td M\implies S=\frac{4\pi Gk_B}{\hbar c}M^2
\end{equation}
Bekenstein-Hawking \footnote{In reality, they presented their result not as a function of the mass $M$, but as a function of the surface area $A$, and thus, the proper Bekenstein-Hawking entropy would be $S_{BH}=\sfrac{c^3k_B}{\hbar G}A$.}

Since black holes are the perfect example of a black body, we could use the Stefan-Boltzmann law for the luminosity $L$
\begin{equation}
	L=-c^2\frac{\td M}{\td t}=\epsilon A \sigma T^4
\end{equation}
where $\sigma=\sfrac{\pi^2 k_B^4}{60\hbar^3c^2}$ its the Stefan-Boltzmann constant, $\epsilon$ is a factor of correction for possible greybody effects
\begin{equation}
	M(t)=M_0\left(1-\frac{t}{t_{BH}}\right)^{\sfrac{1}{3}},\quad t_{BH}\equiv 5120\frac{\pi G^2}{\epsilon\hbar c^4}M_0^3
\end{equation}
stellar black holes (the most numerous type) have masses around $100M_\odot$, meaning a time of evaporation of about $\sim 2.1\cdot10^{73}$ years.

TBD
