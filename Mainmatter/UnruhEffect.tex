\section{Accelerated Observers and Unruh Temperature}
In contrast to previous chapters, in this section there will be no effects of gravitation, and a flat $1+1$ spacetime (for simplicity) will be considered; the difference from the standard flat QFT, here a non inertial (accelerated) observed will be considered. Let an observer measure a self constant two-acceleration $\alpha^\mu\equiv \sfrac{\td^2 x^\mu}{\td\tau^2}$, and thus
\begin{equation}
	\alpha^2=\left(c\frac{\td^2 t}{\td\tau^2}\right)^2-\left(\frac{\td^2x}{\td\tau^2}\right)^2.
\end{equation}
Solutions of such differential equation can be written as
\begin{subequations}
	\begin{gather}
		t(\tau)=t_0+t_1\tau\pm\frac{c}{\alpha}\sinh\left(\frac{\alpha\tau}{c}\right),\quad x(\tau)=x_0+x_1\tau\pm\frac{c^2}{\alpha}\cosh\left(\frac{\alpha\tau}{c}\right),\tag{\theequation \,\,a,b}
	\end{gather}
\end{subequations}
where $t_0,t_1,x_0,x_1$ are constant real parameters. Since the trajectory must ensure that $c^2\td\tau^2=c^2\td t^2-\td x^2$, then the "velocities" must be such that $ct_1=x_1=0$; furthermore, for simplicity we consider some coordinates in which $ct_0=x_0=0$. Such trajectory can be described as the hyperbola
\begin{equation}
	x^2-c^2t^2=\frac{c^2}{\alpha^2}.
\end{equation}
There exists a coordinate system known as Rindler coordinates, that is specially useful in the description of accelerated observers, such coordinates do not map the whole spacetime, and must be divided into two maps: whereas $x>c|t|$ or otherwise. The coordinates used will be named as $(\eta,\xi)$ ($(\tilde\eta,\tilde\xi)$ if working in the second chart), the first could be understood as some sort of temporal coordinate, while the second as a parameter determining the acceleration of the observer.
\begin{wrapfigure}[13]{r}{0.49\textwidth}
	\centering{
		\definecolor{gridcolor}{RGB}{255 216 234} 
\definecolor{gridlabelcolor}{RGB}{255 24 131}

\definecolor{Rightgrid}{RGB}{141, 195, 227} 
\definecolor{RightEx}{RGB}{54, 98, 245}

\definecolor{Leftgrid}{RGB}{222, 198, 247} 
\definecolor{LeftEx}{RGB}{160, 76, 245}
\begin{tikzpicture}[%
	scale=3,%
	GridRight/.style={draw=Rightgrid,thick},%
	SubRight/.style={draw=Rightgrid,thin},%
	GridLeft/.style={draw=Leftgrid,thick},%
	SubLeft/.style={draw=Leftgrid,thin},%
	ExRight/.style={draw=RightEx, thick},%
	ExLeft/.style={draw=LeftEx, thick},%
	tlabels/.style={pos=0.88,above,sloped,yshift=-0.3ex,RightEx},%
	label/.style={%
		postaction={%
			decorate,%
			transform shape,%
			decoration={%
				markings,%
				mark=at position .65 with \node #1;%
			}%
		}%
	},%
	]%
	\pgfmathdeclarefunction{arcosh}{1}{\pgfmathparse{ln(#1+sqrt(#1+1)*sqrt(#1-1))}}
	\pgfmathsetmacro{\Xmax}{1.2}
	\pgfmathsetmacro{\Tmax}{1.2}
	\pgfmathsetmacro{\g}{1}
	\newcommand\mylabelstyle\tiny
	
	% curves t=constant
	\foreach \t in {-3,-2.9375,...,3}{%
		\path[SubRight] (0,0) -- (\Xmax,{\Xmax*tanh(\g*\t)});
	}
	\foreach \t in {-3,-2.75,...,3}{%
		\path[GridRight] (0,0) -- (\Xmax,{\Xmax*tanh(\g*\t)});
	} 
	\foreach \t in {-3,-2.9375,...,3}{%
		\path[SubLeft] (0,0) -- (-\Xmax,{-\Xmax*tanh(\g*\t)});
	}
	\foreach \t in {-3,-2.75,...,3}{%
		\path[GridLeft] (0,0) -- (-\Xmax,{-\Xmax*tanh(\g*\t)});
	}   
	
	% curves x=constant
	\foreach \xx in {0.05,0.1,...,\Xmax}{%
		\path[SubRight]
		plot[domain=-{arcosh(\Xmax/\xx)/\g}:{arcosh(\Xmax/\xx)/\g}]
		({\xx*cosh(\g*\x)},{\xx*sinh(\g*\x)});  
	}
	\foreach \xx in {0.05,0.1,...,\Xmax}{%
		\path[SubLeft]
		plot[domain=-{-arcosh(\Xmax/\xx)/\g}:{-arcosh(\Xmax/\xx)/\g}]
		({-\xx*cosh(\g*\x)},{-\xx*sinh(\g*\x)});  
	}
	\foreach \xx in {0.2,0.4,...,1}{%
		\path[GridRight]
		plot[domain=-{arcosh(\Xmax/\xx)/\g}:{arcosh(\Xmax/\xx)/\g}]
		({\xx*cosh(\g*\x)},{\xx*sinh(\g*\x)});  
	}
	\foreach \xx in {0.2,0.4,...,1}{%
		\path[GridLeft]
		plot[domain=-{-arcosh(\Xmax/\xx)/\g}:{-arcosh(\Xmax/\xx)/\g}]
		({-\xx*cosh(\g*\x)},{-\xx*sinh(\g*\x)});  
	}
	
	\foreach \xx in {0.55}{%
		\path[ExRight]
		plot[domain=-{arcosh(\Xmax/\xx)/\g}:{arcosh(\Xmax/\xx)/\g}]
		({\xx*cosh(\g*\x)},{\xx*sinh(\g*\x)});  
	}
	
	\foreach \xx in {0.55}{%
		\path[ExLeft]
		plot[domain=-{-arcosh(\Xmax/\xx)/\g}:{-arcosh(\Xmax/\xx)/\g}]
		({-\xx*cosh(\g*\x)},{-\xx*sinh(\g*\x)});  
	}
	
	% curve labels
	
	\foreach \xx in {0.75}{%
		\path[label={[above]{\mylabelstyle $\xi=$const}}]
		plot[domain=-{arcosh(\Xmax/\xx)/\g}:{arcosh(\Xmax/\xx)/\g}]
		({\xx*cosh(\g*\x)},{\xx*sinh(\g*\x)});  
	}
	
	\foreach \xx in {0.75}{%
		\path[label={[below]{\mylabelstyle $\tilde \xi=$const}}]
		plot[domain=-{-arcosh(\Xmax/\xx)/\g}:{-arcosh(\Xmax/\xx)/\g}]
		({-\xx*cosh(\g*\x)},{-\xx*sinh(\g*\x)});  
	}
	
	% X-axis, T-axis, and dashed lines t=+/-infty 
	\draw[very thick,-stealth] (-\Xmax,0) -- (\Xmax,0) node[below] {$x$};
	\draw[thick,-stealth] (0,-\Tmax) -- (0,\Tmax) node[left] {$t$};
	\draw[dashed] (0,0) -- (\Xmax,\Tmax) 
	node[pos=0.37,above,sloped,yshift=-.3ex] {\mylabelstyle$\xi=-\infty$}
	node[tlabels,black] {\mylabelstyle$\eta=\infty$};
	\draw[dashed] (0,0) -- (\Xmax,-\Tmax)
	node[tlabels,black] {\mylabelstyle$\eta=-\infty$};
	
	\draw[dashed] (0,0) -- (-\Xmax,-\Tmax) 
	node[tlabels,black] {\mylabelstyle$\tilde\eta=-\infty$};
	\draw[dashed] (0,0) -- (-\Xmax,\Tmax)
	node[pos=0.37,above,sloped,yshift=-.3ex] {\mylabelstyle$\tilde\xi=-\infty$}
	node[tlabels,black] {\mylabelstyle$\tilde\eta=\infty$};
	
\end{tikzpicture}

		\caption{Rindler Coordinates (Left and Right charts).}
		\label{fig: Rindler Coordinates}
	}
\end{wrapfigure}
Such coordinate system can be written\footnote{The choice is not unique, depending on the definition of $\xi$ the exponential factor can be exchanged by a different form, such as $()1+\xi)$ or simply by $\xi$.} as:
\begin{itemize}
	\item Chart $x>c|t|$ (\ref{fig: Rindler Coordinates} blue section),
	\begin{subequations}
		\begin{gather}
			t(\eta,\,\xi)\equiv \frac{c}{\alpha}\sinh\left(\frac{\alpha\eta}{c}\right)e^{\sfrac{\alpha\xi}{c^2}},\tag{\theequation \,\,a}\\ x(\eta,\,\xi)\equiv \frac{c^2}{\alpha}\cosh\left(\frac{\alpha\eta}{c}\right)e^{\sfrac{\alpha\xi}{c^2}},\tag{\theequation \,\,b}
		\end{gather}
	\end{subequations}
	\item Chart $x<c|t|$ (\ref{fig: Rindler Coordinates} purple section),
	\begin{subequations}
		\begin{gather}
			t(\tilde\eta,\,\tilde\xi)\equiv- \frac{c}{\alpha}\sinh\left(\frac{\alpha\tilde\eta}{c}\right)e^{\sfrac{\alpha\tilde\xi}{c^2}},\tag{\theequation \,\,a}\\ x(\tilde\eta,\,\tilde\xi)\equiv -\frac{c^2}{\alpha}\cosh\left(\frac{\alpha\tilde\eta}{c}\right)e^{\sfrac{\alpha\tilde\xi}{c^2}},\tag{\theequation \,\,b}
		\end{gather}
	\end{subequations}
\end{itemize}
One can check that, these coordinates meet the following hyperbolic relation,
\begin{equation}\label{eq: Hyperbolic general acceleration Unruh}
	x^2-c^2t^2=\frac{c^2}{\alpha^2}e^{2\sfrac{\alpha\xi}{c^2}},
\end{equation}
and thus, an observer with some coordinates $(\eta,\xi)$ can be said to experience an acceleration $\alpha e^{-\xi\sfrac{}{c^2}}$. Using such coordinate system, the line element will be\footnote{Note that as it would be expected, the metric is conformally flat.},
\begin{equation}
	c^2\td \tau^2=e^{\sfrac{\alpha\xi}{c^2}}\left[c^2\td\eta^2-\td\xi^2\right].
\end{equation}

The fact that a non-inertial reference system will experience some deviations of a theory in relation to an inertial one should not surprise any reader, since its a basic result of elementary physics, and thus, it is also expected in a relativistic quantum theory to be true. To be able to demonstrate this, we will consider the simplest possible case, a massless and minimally coupled scalar field $\phi$; whose equation of motion given by the Klein-Gordon equation \ref{eq: Klein-Gordon General} will be
\begin{equation}
	e^{-2\sfrac{\alpha\xi}{c^2}}\left[\partial^2_{c\eta}-\partial^2_\xi\right]\phi=0.
\end{equation}
Solutions of such equation are the usual plane waves\footnote{This is a direct result of the fact that, for a $1+1$ scalar theory, conformal symmetry is obtained for $m=\xi=0$.}, just as it would for an inertial observer. Since the solutions of an accelerated reference frame and an inertial observer are the same, it would seem that all phenomena will be described equally by both, but this is not the same, since their metric description will be different, and thus, the field will be related by a Bogoliubov transformation; that is the two observers might disagree on their definition of the vacuum state.

In order to explicitly compute this, it will be useful to use the so called null coordinates $u\equiv c\eta-\xi$ and $v\equiv c\eta+\xi$ for the accelerated observer, and  $U\equiv ct-x$ and $V\equiv ct+x$; through which the solution of the equation of motion will be (depending on the needed chart)
\begin{subequations}\label{eq: Unruh modes}
	\begin{gather}
		\phi_\omega^u\equiv e^{i\omega u\,\hbar^{-1}},\quad \phi_\omega^v\equiv  e^{i\omega v\,\hbar^{-1}},\quad\phi_\omega^{\tilde u}\equiv e^{i\omega \tilde u\,\hbar^{-1}},\quad\phi_\omega^{\tilde v}\equiv  e^{i\omega \tilde v\,\hbar^{-1}}.\tag{\theequation \,\,a-d}
	\end{gather}
\end{subequations}
In addition, null coordinates will also be used for the inertial observer those being $U\equiv ct-x$ and $V\equiv ct+x$; and the relation to the accelerated null coordinates will be
\begin{subequations}\label{eq: Unruh Null coordinates relation}
	\begin{gather}
		U(u,\,v)=-\frac{c^2}{\alpha}e^{-\sfrac{\alpha u}{c^2}},\hspace{1.0cm} V(u,\,v)=\frac{c^2}{\alpha}e^{\sfrac{\alpha v}{c^2}}.\tag{\theequation \,\,a,b}
	\end{gather}
\end{subequations}
Using null coordinates the field can be described as two independent non interactive fields (since there is no autointeration term), that is, $\phi\equiv\phi_u+\phi_v$; in what follows, we will only consider the $\phi_u$ field (defined bellow), but the same can be done for the $\phi_v$ field. Not, lets expand the $\phi_u$ field as a set of Rindler modes \ref{eq: Unruh modes}.a,c meaning
\begin{equation}
	\phi_u\equiv\int_0^\infty \frac{\td \omega}{(2\pi\hbar)\sqrt{2\omega}}\left\{\Theta(-U)\left[a^u_\omega\phi_\omega^u+\bar a^u_\omega\bar \phi_\omega^u \right]+\Theta(U)\left[a^{\tilde u}_\omega\phi_\omega^{\tilde u}+\bar a^{\tilde u}_\omega\bar \phi_\omega^{\tilde u} \right]\right\};
\end{equation}
clearly, it can also be written in terms of planar waves in $U$ coordinates for the inertial observer.

Since we were interested in the difference in vacuum states, we would need to obtain the Bogoliubov coefficients connecting the Rindler modes, with the inertial ones; to do so, we compute the coefficient $\beta_{\Omega,\omega}$ using \ref{eq: Bogoliubov Coefficients General}.b, obtaining
\begin{equation}
	\beta_{\Omega\omega}^u\propto \langle \phi^u_\Omega,\bar\phi_\omega^{U}\rangle=\frac{1}{2\hbar^2}\sqrt{\frac{\Omega}{\omega}}\int_{-\infty}^\infty\exp{\left\{i\left[\omega U(u)+\Omega u\right]\hbar^{-1}\right\}}\td u,
\end{equation}
where the expression for $U(u)$ is given by \ref{eq: Unruh Null coordinates relation}.a. This integral is solvable using the change of variables $z\equiv i\left(\sfrac{\omega\hbar c^2}{\alpha}\right)\exp{\!\left(-\sfrac{a u}{c^2}\right)}$ which will result on
\begin{equation}
	\int_{-\infty}^\infty\exp{\left\{i\left[\omega U(u)+\Omega u\right]\hbar^{-1}\right\}}\td u=\frac{1}{\alpha}\left(-i\frac{\alpha\hbar}{\omega c^2}\right)^{i\sfrac{c^2\Omega}{\alpha\hbar}}\int^\infty_0z^{i\sfrac{c^2\Omega}{\alpha\hbar}-1}e^{-z}\td z;
\end{equation}
where the integral is the Gamma function $\Gamma\left(i\sfrac{c^2\Omega}{\alpha\hbar}\right)$. This means that the Bogoliubov coefficient can be written as
\begin{equation}
	\beta_{\Omega\omega}^u\propto \Gamma\left(i\frac{\Omega c^2}{\alpha\hbar }\right)\sqrt{\frac{\Omega}{\omega}}\left(\frac{c^2\omega}{\alpha\hbar }\right)^{-i\frac{\omega c^2}{\alpha\hbar}}e^{-\frac{\pi\Omega c^2}{2\alpha\hbar}}.
\end{equation}
Now, from \ref{eq: Bogoliubov Number particles general} we know how to compute the number of particles of some momentum $\Omega$ that an accelerated observer would measure in the inertial vacuum; that is
\begin{equation}\label{eq: Bose Distribution Unruh}
	N_\Omega\equiv\langle_U 0|\left(\hat{a}_\Omega^u\right)^\dagger\hat{a}^u_\Omega|_U0\rangle\propto\int_0^\infty\frac{\td\Omega}{(2\pi\hbar)\sqrt{2\Omega}}|\beta^u_{\Omega\Omega}|\propto\frac{1}{e^{\frac{2\pi\Omega c}{\alpha\hbar}}-1}\delta(0).
\end{equation}
This expression has some resemblance with the expected value of particles with energy $\Omega$ for Bose-Einstein statistics; if we were to interpret it as such (which is not an overreach, considering the bosonic nature of the scalar field particles), a temperature might be defined as
\begin{equation}\label{eq: Unruh temp_0}
	T_0\equiv \frac{\alpha\hbar}{2\pi ck_B}.
\end{equation}
This so called temperature though, cannot be interpreted as the temperature that an accelerated Rindler observer would measure (since it is dependent on the coordinate variable $\alpha$), to obtain such value of the temperature, one should consider the Tolman's law\footnote{An sketch of a proof goes as follows: Consider some conserved energy $E_0$ measured by an observer in an stationary gravitational field, such energy relates to the energy measured by another observer by $E_0=\sqrt{g_{00}}E$; since the thermodynamic relation between energy and temperature comes from the entropy $S$ as $T_0=\sfrac{\partial S}{\partial E_0}$, then the proper temperature must follow the relation given by the Tolman's law.}, stating that the proper temperature $T_{\text{Unruh}}$ is given by
\begin{equation}
	T_{\text{Unruh}}=\sqrt{g_{00}}\,T_0=\left(\alpha e^{-\sfrac{\alpha}{c^2}\xi}\right)\frac{\hbar}{2\pi ck_B}\equiv \frac{a\hbar}{2\pi ck_B};
\end{equation}
where $a$ is the acceleration measured by the non-inertial observer, as stated on equation \ref{eq: Hyperbolic general acceleration Unruh}.

\section{Application to Black Holes: Hawking Radiation}
According with the no hair conjecture, black holes can be univocally describe only by three parameters: its mass $M$, its angular momentum $J$ and its electric charge $Q$. For simplicity, lets consider a Schwarzschild black hole\footnote{This is the first found solution of the Einstein field equations \ref{eq: Einstein Field Equations} by Karl Schwarzschild in 1916; Einstein himself was quoted to be amazed by the simplicity and the speed of Schwarzschild's derivation.}, which considers $J=Q=0$; the line element describing the spacetime of such black hole \cite{CanoJones} can be written in spherical coordinates as
\begin{equation}\label{eq: Schwarzschild line element}
	c^2\td\tau^2=\left(1-\frac{2GM}{c^2r}\right)c^2\td t^2-\left(1-\frac{2GM}{c^2r}\right)^{-1}\td r^2-r^2\left(\td\theta^2+\sin^2\theta\td\varphi^2\right),
\end{equation}
where $R_S\equiv\sfrac{2GM}{c^2}$ is the so called \textit{Schwarzschild radius}; the apparent singularity at $r=R_S$ is (as we will show shortly) not a physical one, but a by-product of the coordinate system.

In order to simplify even more the problem (and to be able to create bridge to the previous section), lets consider a $1+1$ Schwarzschild black hole; to obtain such solution, simply take the limit $\td\theta=\td\varphi=0$ at the solution \ref{eq: Schwarzschild line element}. There are two interesting coordinate systems to describe this spacetime; the ''tortoise'' coordinates and the Kruskal–Szekeres coordinates; each with some precise interest. The first of the two comes from the use of the so called tortoise coordinate, given by the expression
\begin{equation}\label{eq: Tortoise coordinate diff}
	\td r^*\equiv \left(1-\frac{R_S}{r}\right)^{-1}\td r,
\end{equation}
meaning that the line element will be
\begin{equation}
	c^2\td\tau^2=\left[1-\frac{R_S}{r\left(r^*\right)}\right]\left[c^2\td t^2-\td r^*\right];
\end{equation}
similarly to the previous section, it is useful to write it using null coordinates $u\equiv ct-r^*$ and $v\equiv ct+r^*$,
\begin{equation}
	c^2\td\tau^2=\left[1-\frac{R_S}{r\left(u,v\right)}\right]\left[\td u-\td v\right].
\end{equation}

Last expression can be rewritten considering that\footnote{To check this, one must use the integrated form of the tortoise coordinate \ref{eq: Tortoise coordinate diff} given by $$r^*(r)=r-R_S+R_S\ln\left(\frac{r}{R_S}-1\right).$$}
\begin{equation}
	1-\frac{R_S}{r\left(u,v\right)}=\frac{R_S}{r}\exp{\left(1-\frac{r}{R_S}\right)}\exp{\left(\frac{v-u}{2R_S}\right)}
\end{equation} 
and, defining the Kruskal–Szekeres null coordinates $(U,\,V)$ as
\begin{subequations}\label{eq: Tortoise-Kruskal relation}
	\begin{gather}
		U\equiv -2R_Se^{-\sfrac{u}{2R_S}},\hspace{1.0cm}V\equiv 2R_Se^{\sfrac{v}{2R_S}},\tag{\theequation \,\,a,b}
	\end{gather}
\end{subequations}
the line element can be written as
\begin{equation}
	c^2\tau^2=\frac{R_S}{r(U,V)}e^{1-\sfrac{r(U,V)}{R_S}}\td U\td V.
\end{equation}
Note that both coordinate systems make the metric conformally flat, and the relation between the null tortoise and null Kruskal–Szekeres coordinates given by \ref{eq: Tortoise-Kruskal relation} is the same as in \ref{eq: Unruh Null coordinates relation} considering $\alpha=\sfrac{c^2}{2R_S}$; therefore the procedure of the previous section can be used here, and thus, according to \ref{eq: Unruh temp_0}, a tortoise observer located at $r\to\infty$ will measure that the black hole emits a thermal bath of massless boson particles at a temperature 
\begin{equation}
	T_{\text{Hawking}}=\frac{\hbar c^3}{8\pi GMk_B}.
\end{equation}
If one where to consider a $3+1$ black hole \cite[sec.\,9.1.4]{QuantumEffects} described by the line element given by \ref{eq: Schwarzschild line element}, the Klein-Gordon equation would not be $\partial^\mu\partial_\nu\phi=0$, but would have an effective potential
\begin{equation}
	\mathcal{V}_{\text{eff}}\equiv\left(1-\frac{R_S}{r}\right)\left[\frac{R_S}{r^3}+\frac{l(l+1)}{r^2}\right],
\end{equation}
$l$ being the quantum orbital angular momentum of the state. The effect of such potential is such that an escaping wave, upon reaching $r\to\infty$ would have changed its frequency $\Omega$, reason for which \ref{eq: Bose Distribution Unruh} would be affected by some greybody factor $\Gamma(\Omega)$ 
\begin{equation}
	N_\Omega\propto\frac{\Gamma(\Omega)}{e^{\sfrac{\Omega}{k_BT_{\text{Hawking}}}}-1}\delta(\mathbf{0}).
\end{equation}
\subsection{Black Hole Thermodynamics}
Once it has been proven that an observer can detect an emission of particles from a black hole, the question of the origin of the required energy for the creation of such particles arises. One can consider that the energy required to create Unruh particles can come from the energy used for the continuous acceleration, but this cannot be the answer for the Hawking radiation, since both observers are free falling. Therefore the only possible answer must be that the particles extract their energy directly from the black hole; meaning that a black hole must be in thermodynamic equilibrium with the field. This reasoning can be used to deduce the evolution of black holes; as stated before, black holes can be described univocally by three parameters, and thus, its entropy fundamental relation can be written as
\begin{equation}
	\td S=\left(\frac{\partial S}{\partial M}\right)\td M+\left(\frac{\partial S}{\partial J}\right)\td J+\left(\frac{\partial S}{\partial Q}\right)\td Q.
\end{equation}
For a Schwarzschild black hole, this relation simplifies (since $Q=J=0$); and considering that the energy must equal $Mc^2$, then one can deduce the so called Bekenstein-Hawking \footnote{In reality, they presented their result not as a function of the mass $M$, but as a function of the surface area $A$, and thus, the proper Bekenstein-Hawking entropy would be $S_{BH}=\sfrac{c^3k_B}{\hbar G}A$.} entropy 
\begin{equation}
	\td S=\frac{c^2}{T_{\text{Hawking}}}\td M\implies S_{\text{BH}}=\frac{4\pi Gk_B}{\hbar c}M^2.
\end{equation}

Since we are considering that the field extracts energy from the black hole, and the energy of the black hole is proportional to its mass; it is then clear that the black hole must be loosing mass. One can easily deduce the temporal mass expression considering the black hole as a black body, and thus following the Stefan-Boltzmann law for the luminosity $L$, i.e.
\begin{equation}
	L\equiv-c^2\frac{\td M}{\td t}=\epsilon A \sigma T^4_{\text{Hawking}}
\end{equation}
where $\sigma$ its the Stefan-Boltzmann constant and $\epsilon$ is a factor of correction for possible greybody effects and deviations from the use of the electromagnetic field (mostly lose of degrees of freedom). Solutions of such differential equation are
\begin{subequations}
	\begin{gather}
		M(t)=M_0\left(1-\frac{t}{t_{BH}}\right)^{\sfrac{1}{3}},\quad t_{BH}\equiv 5120\frac{\pi G^2}{\epsilon\hbar c^4}M_0^3.\tag{\theequation \,\,a,b}
	\end{gather}
\end{subequations}
To get an idea of the strength of such radiation, lets consider an average stellar black hole (the most numerous type), which masses around $100M_\odot$; considering that at $t=0$, a black hole has that mass, it would have lost all of it (it is said to have ``evaporated") in about $\sim 2.1\cdot10^{73}$ years, an unfathomable magnitude comparable to the age of the universe ($13.7\cdot 10^{9}$ years).
