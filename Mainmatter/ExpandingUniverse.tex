The methodology for the analysis of scalar fields in a general manifold was presented in the previous chapter as a  preliminary for the rest of this work, and in particular of the present chapter. It is clear that the presence of symmetries of the theory will simplify computations, and thus, a great start might be an isotropic and homogeneous expanding universe, which is described by the so called Friedmann–Lemaître–Robertson–Walker metric. Using  reduced-circumference polar coordinates, the line element associated with such metric is written as
\begin{subequations}\label{eq: FLRW metric}
	\begin{gather}
		\td l^2 =c^2\td t^2 -a^2(t)\left[\frac{\td r^2}{1-\kappa r^2}+r^2\td\Omega^2\right],\quad \td\Omega\equiv \td\theta^2+\sin^2\varphi\td\varphi^2,\tag{\theequation \,\,a,b}
	\end{gather}
\end{subequations}
where $\kappa$ is the curvature of the space and $a(t)$ is the scale factor determining the expansion.
\section{Expanding scalar field action}
The Weyl tensor associated to the metric presented in \cref{eq: FLRW metric} is identically zero, meaning that the metric is conformally flat, i.e. independently of the space curvature $\kappa$, and therefore there must exist a coordinate system where
\begin{equation}\label{eq: Conformal FLRW}
	\td l^2=a^2(\eta)\eta_{\mu\nu}\td x^\mu\td x^\nu=a^2(\eta)\left[c^2\td \eta^2-\td\mathbf{x}^2\right],
\end{equation}
working in such coordinate system will give the opportunity to use some results of standard scalar field theory. To do so, the action presented in \cref{eq: General scalar field action} will be rewritten in terms of a new field $\rchi(x)\equiv a(\eta)\,\phi(x)$ using the fact that $\sqrt{-g}=a^4$ 
\begin{equation}
	S[\rchi]=\int\frac{1}{2}\left[\partial_\nu\rchi\,\partial^\nu\rchi-\left(\mu^2a^2+\xi R a^2-c^2\frac{a''}{a}\right)\rchi^2-\partial_\eta\left(c^2\rchi^2\frac{a'}{a}\right)\right]\td^4 x,
\end{equation}
where $a'\equiv \partial_\eta a(\eta)$ and equivalently with $a''$.

Dropping the time derivative will result on the following action for the sacalar $\rchi$ field
\begin{equation}
	S[\rchi]=\int\frac{1}{2}\Big[\partial_\nu\rchi\,\partial^\nu\rchi-\left(\mu^2a^2+\xi R a^2-c^2\frac{a''}{a}\right)\rchi^2\Big]\td^4 x,
\end{equation}
being the main source for the current study. In order to obtain the expressions describing the dynamics of this field, the Euler-Lagrange equations in \cref{eq: Euler-Lagrange} will be used, resulting in the generalized Klein-Gordon equation
\begin{subequations}\label{eq: Generalized Klein-Gordon}
	\begin{gather}
		\left[\partial_\nu\partial^\nu+\mu_{\text{eff}}^2(t)\right]\rchi=0,\hspace{1.0cm}\mu_{\text{eff}}^2(t)=\left(\mu^2+\xi R\right)a^2-\frac{a''}{ac^2}.\tag{\theequation \,\,a,b}
	\end{gather}
\end{subequations}

Solutions of the \cref{eq: Generalized Klein-Gordon} are dependent on an integration constant related to the momentum $\mathbf{k}$, and are of the form
\begin{equation}
	\rchi_{\mathbf{k}}(x)=\alpha_\mathbf{k}v_\mathbf{k}(\eta)e^{ -i\mathbf{kx}\hbar^{-1}}+\bar\alpha_\mathbf{k}\bar v_\mathbf{k}(\eta)e^{ i\mathbf{kx}\hbar^{-1}},
\end{equation}
and upon substitution in \eqref{eq: Generalized Klein-Gordon}, one gets the following differential equation
\begin{equation}\label{eq: Expanding diff v}
	v_\mathbf{k}''\,\hbar^2+\omega^2_\mathbf{k}(\eta)\,v_\mathbf{k}=0
\end{equation}
where the dispersion relation $\omega_\mathbf{k}(\eta)$ is defined as,
\begin{equation}\label{eq: Expanding dispersion relation}
	\omega^2_{\mathbf k}(\eta)=\mathbf{k}^2+\hbar^2\mu_{\text{eff}}^2(\eta)=\mathbf{k}^2+\left(m^2c^2+\xi\hbar^2 R\right)a^2-\hbar^2\frac{a''}{ac^2}.
\end{equation}

Solving \cref{eq: Expanding diff v} will in turn give the form of the set of solutions $\left\{\rchi_{\mathbf{k}}\right\}$ needed to describe the general expression of $\rchi(x)$; since we are currently considering general expansion parameters and curvature scalars, the following computations will be made using a general set of $v_\mathbf{k}$ functions. This functions nevertheless have some interesting properties, such as being capable of a choice of normalization, and a constant of motion: the imaginary part of $v_\mathbf{k}\bar{v}_\mathbf{k}'$. Lets check the last statement
\begin{equation}
	\frac{\partial}{\partial \eta }\text{Im}(v_\mathbf{k}\bar{v}_\mathbf{k}')=\frac{\partial}{\partial \eta}\left(\frac{v_\mathbf{k}\bar{v}_\mathbf{k}'-\bar v_\mathbf{k}v'_\mathbf{k}}{2i}\right)=\frac{v_\mathbf{k}\bar v_\mathbf{k}'*-\bar v_\mathbf{k}v_\mathbf{k}''}{2i}=0
\end{equation}
last step is result from dispersion relation. Since the functions $v_\mathbf{k}$ are capable to a choice in normalization, we will choose a set of solutions of \cref{eq: Expanding diff v} such that $\text{Im}(v_\mathbf{k}\bar{v}_\mathbf{k}')$ its independent of the momentum $\mathbf{k}$, and equal for all modes, this constant of motion will simply be defined as
\begin{equation}
	\text{Im}(v\bar{v}')\equiv \text{Im}(v_\mathbf{k}\bar{v}_\mathbf{k}'),\quad \forall\,\mathbf{k}.
\end{equation}

The most general solution $\rchi(x)$ of equation \cref{eq: Generalized Klein-Gordon} can be written as a Fourier mode expansion
\begin{equation}
	\rchi(x)=\int\frac{\td^3\mathbf{k}}{(2\pi\hbar)^3}\left[a_\mathbf{k}v_\mathbf{k}(\eta)e^{-i\mathbf{kx}\hbar^{-1}}\!\!+\bar a_\mathbf{k}\bar v_\mathbf{k}(\eta)e^{i\mathbf{kx}\hbar^{-1}}\right]
\end{equation}

\section{Quantization}
\begin{subequations}\label{eq: a commutators expanding}
	\begin{gather}
		[\hat{a}_\mathbf{k},\hat{a}_\mathbf{q}^\dagger]=\frac{(2\pi\hbar)^3\hbar c}{2\text{Im}(v\bar v')}\delta^3(\mathbf{k}-\mathbf{q})\,,\hspace{1.0cm}[\hat{a}_\mathbf{k},\hat{a}_\mathbf{q}]=[\hat{a}_\mathbf{k}^\dagger,\hat{a}_\mathbf{q}^\dagger]=0\tag{\theequation \,\,a-c}
	\end{gather}
\end{subequations}
\begin{multline}
	\left[\hat{\rchi}(\mathbf{x}),\,\hat{\Pi}(\mathbf{y})\right]=\frac{1}{c}\int\frac{\td^3\mathbf{k}\td^3\mathbf{q}}{(2\pi\hbar)^6}\left\{\left[\hat{a}_\mathbf{k},\hat{a}_\mathbf{q}\right]v_\mathbf{k}v'_\mathbf{q}e^{-i\left(\mathbf{kx}+\mathbf{qy}\right)\hbar^{-1}}+\left[\hat{a}_\mathbf{k}^\dagger, \hat{a}_\mathbf{q}^\dagger\right]\bar v_\mathbf{k}\bar v_\mathbf{q}'e^{-i(\mathbf{kx}-\mathbf{qy})\hbar^{-1}}\right.+\\
	+\left.\left[\hat{a}_\mathbf{k},\hat{a}_\mathbf{q}^\dagger\right] v_\mathbf{k}\bar v'_\mathbf{q}e^{-i(\mathbf{kx}-\mathbf{qy})\hbar^{-1}}-\left[\hat{a}_\mathbf{q},\hat{a}_\mathbf{k}^\dagger\right]\bar v_\mathbf{k}v_\mathbf{q}'e^{i(\mathbf{kx}-\mathbf{qy})\hbar^{-1}}\right\}
\end{multline}
using \cref{eq: a commutators expanding} and considering that the proportional factor of \ref{eq: a commutators expanding}.a to be $\alpha$
\begin{equation}
	\left[\hat{\rchi}(\mathbf{x}),\,\hat{\Pi}(\mathbf{y})\right]=\frac{\alpha}{c}\int\frac{\td^3\mathbf{k}}{(2\pi\hbar)^6}2i\text{Im}(v_\mathbf{k}\bar v_\mathbf{k}')e^{-i(\mathbf{kx}-\mathbf{qy})\hbar^{-1}}
\end{equation}
since $\text{Im}(v_\mathbf{k}\bar v_\mathbf{k}')$ was considered to be momentum independent,
\begin{equation}
	\left[\hat{\rchi}(\mathbf{x}),\,\hat{\Pi}(\mathbf{y})\right]=i\frac{2\alpha\text{Im}(v\bar v')}{c(2\pi\hbar)^3}\delta^3(\mathbf{x}-\mathbf{y})
\end{equation}
and, from equation \ref{eq: General canonical commutator}.a one can solve for $\alpha$, resulting in the value present in equation \ref{eq: a commutators expanding}.




Let $\xi=0$, i.e. work in the nonminimally coupled scalar field; then the Hamiltonian will be
\begin{equation}
	\hat{\mathcal{H}}(t)=\int\frac{c}{2}\left[\hat{\Pi}^2+\left(\mathbf{\nabla}\hat{\rchi}\right)^2+\mu_{\text{eff}}^2(t)\hat{\rchi}^2\right]\td^3\mathbf{x}
\end{equation}
\begin{comment}
	\begin{multline}
		\hat{\Pi}^2=\frac{1}{c^2}\int\frac{\td^3\mathbf{k}\td^3\mathbf{q}}{(2\pi\hbar)^6}\left[\hat{a}_\mathbf{k}\hat{a}_\mathbf{q}v'_\mathbf{k}v'_\mathbf{q}e^{-i(\mathbf{k}+\mathbf{q})\mathbf{x}\hbar^{-1}}+\hat{a}_\mathbf{k}\hat{a}_\mathbf{q}^\dagger v'_\mathbf{k}v^{'*}_\mathbf{q}e^{-i(\mathbf{k}-\mathbf{q})\mathbf{x}\hbar^{-1}}\right.+\\
		+\left.\hat{a}^\dagger_\mathbf{k}\hat{a}_\mathbf{q}v^{'*}_\mathbf{k}v'_\mathbf{q}e^{i(\mathbf{k}-\mathbf{q})\mathbf{x}\hbar^{-1}}+\hat{a}_\mathbf{k}^\dagger\hat{a}_\mathbf{q}^\dagger v^{'*}_\mathbf{k}v^{'*}_\mathbf{q}e^{i(\mathbf{k}+\mathbf{q})\mathbf{x}\hbar^{-1}}\right]
	\end{multline}
	\begin{multline}
		\left(\mathbf{\nabla}\hat{\rchi}\right)^2=-\frac{1}{\hbar^2}\int\frac{\td^3\mathbf{k}\td^3\mathbf{q}}{(2\pi\hbar)^6}\mathbf{k}\mathbf{q}\left[\hat{a}_\mathbf{k}\hat{a}_\mathbf{q}v_\mathbf{k}v_\mathbf{q}e^{-i(\mathbf{k}+\mathbf{q})\mathbf{x}\hbar^{-1}}-\hat{a}_\mathbf{k}\hat{a}_\mathbf{q}^\dagger v_\mathbf{k}v^*_\mathbf{q}e^{-i(\mathbf{k}-\mathbf{q})\mathbf{x}\hbar^{-1}}-\right.\\
		-\left.\hat{a}_\mathbf{k}^\dagger\hat{a}_\mathbf{q}v_\mathbf{k}^*v_\mathbf{q}e^{i(\mathbf{k}-\mathbf{q})\mathbf{x}\hbar^{-1}}+\hat{a}^\dagger_\mathbf{k}\hat{a}^\dagger_\mathbf{q}v^*_\mathbf{k}v^*_\mathbf{q}e^{i(\mathbf{k}+\mathbf{q})\mathbf{x}\hbar^{-1}}\right]
	\end{multline}
	\begin{multline}
		\hat{\rchi}^2=\int\frac{\td^3\mathbf{k}\td^3\mathbf{q}}{(2\pi\hbar)^6}\left[\hat{a}_\mathbf{k}\hat{a}_\mathbf{q}v_\mathbf{k}v_\mathbf{q}e^{-i(\mathbf{k}+\mathbf{q})\mathbf{x}\hbar^{-1}}+\hat{a}_\mathbf{k}\hat{a}_\mathbf{q}^\dagger v_\mathbf{k}v^*_\mathbf{q}e^{-i(\mathbf{k}-\mathbf{q})\mathbf{x}\hbar^{-1}}+\right.\\
		+\left.\hat{a}_\mathbf{k}^\dagger\hat{a}_\mathbf{q}v_\mathbf{k}^*v_\mathbf{q}e^{i(\mathbf{k}-\mathbf{q})\mathbf{x}\hbar^{-1}}+\hat{a}^\dagger_\mathbf{k}\hat{a}^\dagger_\mathbf{q}v^*_\mathbf{k}v^*_\mathbf{q}e^{i(\mathbf{k}+\mathbf{q})\mathbf{x}\hbar^{-1}}\right]
	\end{multline}
	\begin{multline}
		\hat{\mathcal{H}}=\frac{c}{2}\int\frac{\td^3\mathbf{k}\td^3\mathbf{q}}{(2\pi\hbar)^3}\left\{\hat{a}_\mathbf{k}\hat{a}_\mathbf{q}\left[\frac{1}{c^2}v'_\mathbf{k}v'_\mathbf{q}-\left(\frac{1}{\hbar^2}\mathbf{k}\mathbf{q}-\mu^2_{\text{eff}}\right)v_\mathbf{k}v_\mathbf{q}\right]\delta^3(\mathbf{k}+\mathbf{q})+\right.\\
		+\left.\hat{a}_\mathbf{k}\hat{a}^\dagger_\mathbf{q}\left[\frac{1}{c^2}v'_\mathbf{k}v^{'*}_\mathbf{q}+\left(\frac{1}{\hbar^2}\mathbf{k}\mathbf{q}+\mu^2_{\text{eff}}\right)v_\mathbf{k}v^*_\mathbf{q}\right]\delta^3(\mathbf{k}-\mathbf{q})+\right.\\
		+\left.\hat{a}^\dagger_\mathbf{k}\hat{a}_\mathbf{q}\left[\frac{1}{c^2}v^{'*}_\mathbf{k}v^{'}_\mathbf{q}+\left(\frac{1}{\hbar^2}\mathbf{k}\mathbf{q}+\mu^2_{\text{eff}}\right)v^*_\mathbf{k}v_\mathbf{q}\right]\delta^3(\mathbf{k}-\mathbf{q})+\right.\\
		+\left.\hat{a}^\dagger_\mathbf{k}\hat{a}^\dagger_\mathbf{q}\left[\frac{1}{c^2}v^{'*}_\mathbf{k}v^{'*}_\mathbf{q}-\left(\frac{1}{\hbar^2}\mathbf{k}\mathbf{q}-\mu^2_{\text{eff}}\right)v^*_\mathbf{k}v^*_\mathbf{q}\right]\delta^3(\mathbf{k}+\mathbf{q})\right\}
	\end{multline}
	\begin{multline}
		\hat{\mathcal{H}}=\frac{c}{2}\int\frac{\td^3\mathbf{k}}{(2\pi\hbar)^3}\left\{\hat{a}_\mathbf{k}\hat{a}_{-\mathbf{k}}\left[\frac{1}{c^2}v'_\mathbf{k}v'_\mathbf{k}+\frac{1}{\hbar^2}\omega^2_\mathbf{k}(t)v_\mathbf{k}v_\mathbf{k}\right]+\right.\\
		+\left.\hat{a}_\mathbf{k}\hat{a}^\dagger_\mathbf{k}\left[\frac{1}{c^2}v'_\mathbf{k}v^{'*}_\mathbf{k}+\frac{1}{\hbar^2}\omega^2_\mathbf{k}(t)v_\mathbf{k}v^*_\mathbf{k}\right]+\right.\\
		+\left.\hat{a}^\dagger_\mathbf{k}\hat{a}_\mathbf{k}\left[\frac{1}{c^2}v^{'*}_\mathbf{k}v^{'}_\mathbf{k}+\frac{1}{\hbar^2}\omega^2_\mathbf{k}(t)v^*_\mathbf{k}v_\mathbf{k}\right]+\right.\\
		+\left.\hat{a}^\dagger_\mathbf{k}\hat{a}^\dagger_{-\mathbf{k}}\left[\frac{1}{c^2}v^{'*}_\mathbf{k}v^{'*}_\mathbf{k}+\frac{1}{\hbar^2}\omega^2_\mathbf{k}(t)v^*_\mathbf{k}v^*_\mathbf{k}\right]\right\}
	\end{multline}
\end{comment}

\begin{equation}
	\hat{\mathcal{H}}=\frac{c}{2}\int\frac{\td^3\mathbf{k}}{(2\pi\hbar)^3}\left[\hat{a}_\mathbf{k}\hat{a}_{-\mathbf{k}}F_\mathbf{k}+\hat{a}^\dagger_\mathbf{k}\hat{a}^\dagger_{-\mathbf{k}}\bar F_\mathbf{k}+\left(2\hat{a}_\mathbf{k}^\dagger\hat{a}_\mathbf{k}+\frac{(2\pi\hbar)^3\hbar c}{2\text{Im}(v\bar v')}\delta^3(\mathbf{0})\right)E_\mathbf{k}\right]
\end{equation}
where
\begin{subequations}
	\begin{gather}
		F_\mathbf{k}(t)=\left(\frac{1}{\hbar c}\right)^2\left[\hbar^2v^{'2}_\mathbf{k}+\omega^2_\mathbf{k}(t)\,c^2 v_\mathbf{k}^2\right],\quad E_\mathbf{k}(t)=\left(\frac{1}{\hbar c}\right)^2\left[\hbar^2\big|v'_\mathbf{k}\big|^2+\omega^2_\mathbf{k}(t)\,c^2 \big|v_\mathbf{k}\big|^2\right].\tag{\theequation \,\,a,b}
	\end{gather}
\end{subequations}
\section{Instantaneous Vacuum State}
Note that the only way a vacuum state $|0\rangle$ could remain an eigenstate of the Hamiltonian at all times, would be if $F_\mathbf{k}=0$, which would mean
\begin{equation}
	F_\mathbf{k}(t)=\left(\frac{1}{\hbar c}\right)^2\left[\hbar^2v^{'2}_\mathbf{k}+\omega^2_\mathbf{k}(t)\,c^2 v_\mathbf{k}^2\right]=0,
\end{equation}
solving for $v_\mathbf{k}$ gives the following expression
\begin{equation}
	v_\mathbf{k}(t)=\text{C}\exp{\left[\pm \frac{c}{i\hbar}\int\omega_\mathbf{k}\left(\eta\right)\td \eta\right]},
\end{equation}
which is not compatible with \ref{eq: Expanding diff v} except for a time independent dispersion relation $\omega_\mathbf{k}$.

The last result implies that, at different times one can (and should) define different vacuum states; and thus, we will define the \textit{instantaneous vacuum state} $|_{(v)}0\rangle$ as the one that at some time $t_0$ will minimize the energy density. Since all possible states are related by Bogolyubov transformations, finding the instantaneous vacumm state is the same as finding the set of functions $v_\mathbf{f}$ that are simultaneously solution of \ref{eq: Expanding diff v} and minimize
\begin{equation}
	\langle_{(v)}0|\hat{\mathcal{H}}(t_0)|_{(v)}0\rangle=\rho(t_0)\delta^3(\mathbf{0})=\frac{\hbar c^2 \,\delta^3(\mathbf{0})}{4\text{Im}(v\bar v')}\int\td^3\mathbf{k}\,E_\mathbf{k}
\end{equation}
To minimise the energy density of the vacuum state is to find the set of functions $v_\mathbf{k}$ that minimise $E_\mathbf{k}$. Suppose that $v_\mathbf{k}$ can be written as
\begin{equation}
	v_\mathbf{k}=r_\mathbf{k}e^{i\alpha_\mathbf{k}}
\end{equation}
since Im$(v\bar v')$ was constant through time
\begin{equation}
	r_\mathbf{k}^2\alpha'_\mathbf{k}=-\text{Im}(v\bar v')
\end{equation}
this means
\begin{equation}
	E_\mathbf{k}=\left(\frac{1}{\hbar c}\right)^2\left\{\hbar^2\left[r^{'2}_\mathbf{k}+\text{Im}^2\left(v\bar v'\right)\frac{1}{r_\mathbf{k}^2}\right]+\omega^2_\mathbf{k}\,c^2r_\mathbf{k}^2\right\}
\end{equation}
the minimum of this function must fulfil $r'_\mathbf{k}(t_0)=0$. Now, if $\omega_\mathbf{k}^2(t_0)$ and $\text{Im}(v\bar v')$ have the same sign, the minimum of $E_\mathbf{k}$ happens when $r_\mathbf{k}(t_0)=\left[\frac{\hbar\text{Im}(v\bar v')}{\omega_\mathbf{k}(t_0)\,c}\right]^{1/2}$.

If there is a minimum, then
\begin{equation}
	v_\mathbf{k}(t_0)=\left[\frac{\hbar\text{Im}(v\bar v')}{\omega_\mathbf{k}(t_0)\,c}\right]^{1/2}\!\!\!\!e^{i\alpha_\mathbf{k}(t_0)}\hspace{1.0cm}v'_\mathbf{k}(t_0)=-c\frac{\omega_\mathbf{k}(t_0)}{ih}\,v_\mathbf{k}(t_0)
\end{equation}
under these functions,
\begin{equation}
	E_\mathbf{k}(t_0)=2\frac{\text{Im}(v\bar v')}{\hbar c}\omega_\mathbf{k}(t_0)\hspace{1.0cm}F_\mathbf{k}(t_0)=0
\end{equation}
meaning
\begin{equation}
	\hat{\mathcal{H}}(t_0)=\text{Im}(v\bar v')\frac{1}{\hbar}\int\frac{\td^3\mathbf{k}}{(2\pi\hbar)^3}\left(2\hat{a}_\mathbf{k}^\dagger\hat{a}_\mathbf{k}+\frac{(2\pi\hbar)^3\hbar c}{2\text{Im}(v\bar v')}\delta^3(\mathbf{0})\right)\omega_\mathbf{k}(t_0)
\end{equation}
which is equivalent to the standard Hamiltonian for a scalar field without the presence of gravity.
\paragraph{Bogolyubov Transformation}

The expression of the field $\rchi$ at two different times must be related to a Bogoliubov transformation, with coefficients
\begin{equation}
	\alpha_{\mathbf{kp}}=\frac{(2\pi\hbar)^3\hbar c}{2\text{Im}(v\bar v')}\langle \rchi_\mathbf{k}(t_0),\,\rchi_\mathbf{p}(t)\rangle\hspace{1.0cm}\beta_{\mathbf{kp}}=-\frac{(2\pi\hbar)^3\hbar c}{2\text{Im}(v\bar v')}\langle \rchi_\mathbf{k}(t_0),\,\bar \rchi_\mathbf{p}(t)\rangle
\end{equation}
since the field can be written as $\rchi_\mathbf{k}=v_\mathbf{k}e^{i\mathbf{kx}\sfrac{}{h}}$ from the expression of the inner product one can see that
\begin{equation}
	\alpha_{\mathbf{kp}}\propto \delta^3(\mathbf{k}-\mathbf{p})\hspace{1.0cm}\beta_{\mathbf{kp}}\propto \delta^3(\mathbf{k}+\mathbf{p})
\end{equation}
therefore it is possible to write
\begin{equation}
	v_\mathbf{k}(t)=\alpha_\mathbf{k}v_\mathbf{k}(t_0)+\beta_\mathbf{k}\bar v_\mathbf{k}(t_0)
\end{equation}
where, recalling that $\text{Im}(v\bar v')$ is constant through time, 
\begin{equation}
	|\alpha_\mathbf{k}|^2-|\beta_\mathbf{k}|^2=1
\end{equation}
To obtain the value of $\langle_{(t_0)}0|\hat{\mathcal{H}}(t)|_{(t_0)}0\rangle$ lets first compute


\begin{equation}
	\langle_{(t_0)}0|\hat{\mathcal{N}}_\mathbf{k}^{(a)}(t)|_{(t_0)}0\rangle=\langle_{(t_0)}0|\hat{a}^\dagger_\mathbf{k}(t)\hat{a}_\mathbf{k}(t)|_{(t_0)}0\rangle=\Big|\beta_\mathbf{k}\Big|^2\frac{(2\pi\hbar)^3\hbar c}{2\text{Im}(v\bar v')}\delta^3(\mathbf{0})
\end{equation}
therefore
\begin{equation}
	\langle_{(t_0)}0|\hat{\mathcal{H}}(t)|_{(t_0)}0\rangle=\delta^3(\mathbf{0})\int\td^3\mathbf{k}\left(\frac{1}{2}+\Big|\beta_\mathbf{k}\Big|^2\right)c\,\omega_\mathbf{k}(t)\geq \langle_{(t_0)}0|\hat{\mathcal{H}}(t_0)|_{(t_0)}0\rangle
\end{equation}
meaning, if $\beta_\mathbf{k}\not=0$ for all $\mathbf{k}$ then, at a time $t>t_0$ the energy density will be different in relation to the original vacuum.