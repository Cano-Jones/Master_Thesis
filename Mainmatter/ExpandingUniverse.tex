The methodology for the analysis of scalar fields in a general manifold was presented in the previous chapter as a  preliminary for the rest of this work, and in particular of the present chapter. It is clear that the presence of symmetries of the theory will simplify computations, and thus, a great start might be an isotropic and homogeneous expanding universe, which is described by the so called Friedmann–Lemaître–Robertson–Walker metric. Using  reduced-circumference polar coordinates, the line element associated with such metric is written as
\begin{subequations}\label{eq: FLRW metric}
	\begin{gather}
		\td l^2 =c^2\td t^2 -a^2(t)\left[\frac{\td r^2}{1-\kappa r^2}+r^2\td\Omega^2\right],\quad \td\Omega\equiv \td\theta^2+\sin^2\varphi\td\varphi^2,\tag{\theequation \,\,a,b}
	\end{gather}
\end{subequations}
where $\kappa$ is the curvature of the space and $a(t)$ is the scale factor determining the expansion. The associated curvature scalar $R$ is given by
\begin{equation}\label{eq: FLRW Curvature}
	R=\frac{6}{c^2}\left[\frac{\ddot{a}}{a}+\left(\frac{\dot{a}}{a}\right)^2\right],
\end{equation}
needed to the coupling of the field with gravity as previously stated.
\section{Expanding scalar field action}
The Weyl tensor associated to the metric presented in \cref{eq: FLRW metric} is identically zero, meaning that the metric is conformally flat, i.e. independently of the space curvature $\kappa$, and therefore there must exist a coordinate system where
\begin{equation}\label{eq: Conformal FLRW}
	\td l^2=a^2(\eta)\eta_{\mu\nu}\td x^\mu\td x^\nu=a^2(\eta)\left[c^2\td \eta^2-\td\mathbf{x}^2\right],
\end{equation}
working in such coordinate system will give the opportunity to use some results of standard scalar field theory. To do so, the action presented in \cref{eq: General scalar field action} will be rewritten in terms of a new field $\rchi(x)\equiv a(\eta)\,\phi(x)$ using the fact that $\sqrt{-g}=a^4$ 
\begin{equation}
	S[\rchi]=\int\frac{1}{2}\left[\partial_\nu\rchi\,\partial^\nu\rchi-\left(\mu^2a^2+\xi R a^2-c^2\frac{a''}{a}\right)\rchi^2-\partial_\eta\left(c^2\rchi^2\frac{a'}{a}\right)\right]\td^4 x,
\end{equation}
where $a'\equiv \partial_\eta a(\eta)$ and equivalently with $a''$.

Dropping the time derivative will result on the following action for the sacalar $\rchi$ field
\begin{equation}
	S[\rchi]=\int\frac{1}{2}\Big[\partial_\nu\rchi\,\partial^\nu\rchi-\left(\mu^2a^2+\xi R a^2-c^2\frac{a''}{a}\right)\rchi^2\Big]\td^4 x,
\end{equation}
being the main source for the current study. In order to obtain the expressions describing the dynamics of this field, the Euler-Lagrange equations in \cref{eq: Euler-Lagrange} will be used, resulting in the generalized Klein-Gordon equation
\begin{subequations}\label{eq: Generalized Klein-Gordon}
	\begin{gather}
		\left[\partial_\nu\partial^\nu+\mu_{\text{eff}}^2(t)\right]\rchi=0,\hspace{1.0cm}\mu_{\text{eff}}^2(t)=\left(\mu^2+\xi R\right)a^2-\frac{a''}{ac^2}.\tag{\theequation \,\,a,b}
	\end{gather}
\end{subequations}

Solutions of the \cref{eq: Generalized Klein-Gordon} are dependent on an integration constant related to the momentum $\mathbf{k}$, and are of the form
\begin{equation}
	\rchi_{\mathbf{k}}(x)=\alpha_\mathbf{k}v_\mathbf{k}(\eta)e^{ -i\mathbf{kx}\hbar^{-1}}+\bar\alpha_\mathbf{k}\bar v_\mathbf{k}(\eta)e^{ i\mathbf{kx}\hbar^{-1}},
\end{equation}
and upon substitution in \eqref{eq: Generalized Klein-Gordon}, one gets the following differential equation
\begin{equation}\label{eq: Expanding diff v}
	v_\mathbf{k}''\,\hbar^2+\omega^2_\mathbf{k}(\eta)\,v_\mathbf{k}=0
\end{equation}
where the dispersion relation $\omega_\mathbf{k}(\eta)$ is defined as,
\begin{equation}\label{eq: Expanding dispersion relation}
	\omega^2_{\mathbf k}(\eta)=\mathbf{k}^2+\hbar^2\mu_{\text{eff}}^2(\eta)=\mathbf{k}^2+\left(m^2c^2+\xi\hbar^2 R\right)a^2-\hbar^2\frac{a''}{ac^2}.
\end{equation}

Solving \cref{eq: Expanding diff v} will in turn give the form of the set of solutions $\left\{\rchi_{\mathbf{k}}\right\}$ needed to describe the general expression of $\rchi(x)$; since we are currently considering general expansion parameters and curvature scalars, the following computations will be made using a general set of $v_\mathbf{k}$ functions. This functions nevertheless have some interesting properties, such as being capable of a choice of normalization, and a constant of motion: the imaginary part of $v_\mathbf{k}\bar{v}_\mathbf{k}'$. Lets check the last statement
\begin{equation}
	\frac{\partial}{\partial \eta }\text{Im}(v_\mathbf{k}\bar{v}_\mathbf{k}')=\frac{\partial}{\partial \eta}\left(\frac{v_\mathbf{k}\bar{v}_\mathbf{k}'-\bar v_\mathbf{k}v'_\mathbf{k}}{2i}\right)=\frac{v_\mathbf{k}\bar v_\mathbf{k}'-\bar v_\mathbf{k}v_\mathbf{k}''}{2i}=0
\end{equation}
last step is result from dispersion relation. Since the functions $v_\mathbf{k}$ are capable to a choice in normalization, we will choose a set of solutions of \cref{eq: Expanding diff v} such that $\text{Im}(v_\mathbf{k}\bar{v}_\mathbf{k}')$ its independent of the momentum $\mathbf{k}$, and equal for all modes, this constant of motion will simply be defined as
\begin{equation}
	\text{Im}(v\bar{v}')\equiv \text{Im}(v_\mathbf{k}\bar{v}_\mathbf{k}'),\quad \forall\,\mathbf{k}.
\end{equation}

The most general solution $\rchi(x)$ of equation \cref{eq: Generalized Klein-Gordon} can be written as a Fourier mode expansion
\begin{equation}\label{eq: Expanding General Solution}
	\rchi(x)=\int\frac{\td^3\mathbf{k}}{(2\pi\hbar)^3}\left[a_\mathbf{k}v_\mathbf{k}(\eta)e^{-i\mathbf{kx}\hbar^{-1}}\!\!+\bar a_\mathbf{k}\bar v_\mathbf{k}(\eta)e^{i\mathbf{kx}\hbar^{-1}}\right].
\end{equation}

\section{Quantization}
After promoting the fields to operators, and imposing commutation relations, one gets the following rules for the creation and annihilation operators from equations \ref{eq: general a operator commutator rules}
\begin{subequations}\label{eq: a commutators expanding}
	\begin{gather}
		[\hat{a}_\mathbf{k},\hat{a}_\mathbf{q}^\dagger]=\frac{(2\pi\hbar)^3\hbar c}{2\text{Im}(v\bar v')}\delta^3(\mathbf{k}-\mathbf{q})\,,\hspace{1.0cm}[\hat{a}_\mathbf{k},\hat{a}_\mathbf{q}]=[\hat{a}_\mathbf{k}^\dagger,\hat{a}_\mathbf{q}^\dagger]=0\tag{\theequation \,\,a-c},
	\end{gather}
\end{subequations}
to be able to verify the value of the proportional factor in \ref{eq: a commutators expanding}.a, lets compute the commutator between $\hat\rchi$ and $\hat\Pi$, i.e.
\begin{multline}
	\left[\hat{\rchi}(\mathbf{x}),\,\hat{\Pi}(\mathbf{y})\right]=\frac{1}{c}\int\frac{\td^3\mathbf{k}\td^3\mathbf{q}}{(2\pi\hbar)^6}\left\{\left[\hat{a}_\mathbf{k},\hat{a}_\mathbf{q}\right]v_\mathbf{k}v'_\mathbf{q}e^{-i\left(\mathbf{kx}+\mathbf{qy}\right)\hbar^{-1}}+\left[\hat{a}_\mathbf{k}^\dagger, \hat{a}_\mathbf{q}^\dagger\right]\bar v_\mathbf{k}\bar v_\mathbf{q}'e^{-i(\mathbf{kx}-\mathbf{qy})\hbar^{-1}}\right.+\\
	+\left.\left[\hat{a}_\mathbf{k},\hat{a}_\mathbf{q}^\dagger\right] v_\mathbf{k}\bar v'_\mathbf{q}e^{-i(\mathbf{kx}-\mathbf{qy})\hbar^{-1}}-\left[\hat{a}_\mathbf{q},\hat{a}_\mathbf{k}^\dagger\right]\bar v_\mathbf{k}v_\mathbf{q}'e^{i(\mathbf{kx}-\mathbf{qy})\hbar^{-1}}\right\}
\end{multline}
using expresions \ref{eq: a commutators expanding} and considering that the proportional factor of \ref{eq: a commutators expanding}.a to be $\alpha$, previous expression simplify to the following one,
\begin{equation}
	\left[\hat{\rchi}(\mathbf{x}),\,\hat{\Pi}(\mathbf{y})\right]=\frac{\alpha}{c}\int\frac{\td^3\mathbf{k}}{(2\pi\hbar)^6}2i\text{Im}(v_\mathbf{k}\bar v_\mathbf{k}')e^{-i(\mathbf{kx}-\mathbf{qy})\hbar^{-1}};
\end{equation}
since $\text{Im}(v_\mathbf{k}\bar v_\mathbf{k}')$ was considered to be momentum independent, this implies that
\begin{equation}
	\left[\hat{\rchi}(\mathbf{x}),\,\hat{\Pi}(\mathbf{y})\right]=i\frac{2\alpha\text{Im}(v\bar v')}{c(2\pi\hbar)^3}\delta^3(\mathbf{x}-\mathbf{y}),
\end{equation}
and, from equation \ref{eq: General canonical commutator}.a one can solve for $\alpha$, resulting in the value present in equation \ref{eq: a commutators expanding}.

\vspace*{0.25cm}
The next step in the quantization procedure is to obtain the Hamiltonian $\hat{\mathcal{H}}$ that spans the Fock space; to do so, we use the definition in equation \ref{eq: General Hamiltonian} alongside the energy momentum tensor \ref{eq: Energy-Momentum scalar}. As a simplification, lets consider a minimally coupled theory, i.e. $\xi=0$; then the Hamiltonian will be
\begin{equation}
	\hat{\mathcal{H}}(t)=\int\frac{c}{2}\left[\hat{\Pi}^2+\left(\mathbf{\nabla}\hat{\rchi}\right)^2+\mu_{\text{eff}}^2(t)\hat{\rchi}^2\right]\td^3\mathbf{x}.
\end{equation}
\begin{comment}
	\begin{multline}
		\hat{\Pi}^2=\frac{1}{c^2}\int\frac{\td^3\mathbf{k}\td^3\mathbf{q}}{(2\pi\hbar)^6}\left[\hat{a}_\mathbf{k}\hat{a}_\mathbf{q}v'_\mathbf{k}v'_\mathbf{q}e^{-i(\mathbf{k}+\mathbf{q})\mathbf{x}\hbar^{-1}}+\hat{a}_\mathbf{k}\hat{a}_\mathbf{q}^\dagger v'_\mathbf{k}v^{'*}_\mathbf{q}e^{-i(\mathbf{k}-\mathbf{q})\mathbf{x}\hbar^{-1}}\right.+\\
		+\left.\hat{a}^\dagger_\mathbf{k}\hat{a}_\mathbf{q}v^{'*}_\mathbf{k}v'_\mathbf{q}e^{i(\mathbf{k}-\mathbf{q})\mathbf{x}\hbar^{-1}}+\hat{a}_\mathbf{k}^\dagger\hat{a}_\mathbf{q}^\dagger v^{'*}_\mathbf{k}v^{'*}_\mathbf{q}e^{i(\mathbf{k}+\mathbf{q})\mathbf{x}\hbar^{-1}}\right]
	\end{multline}
	\begin{multline}
		\left(\mathbf{\nabla}\hat{\rchi}\right)^2=-\frac{1}{\hbar^2}\int\frac{\td^3\mathbf{k}\td^3\mathbf{q}}{(2\pi\hbar)^6}\mathbf{k}\mathbf{q}\left[\hat{a}_\mathbf{k}\hat{a}_\mathbf{q}v_\mathbf{k}v_\mathbf{q}e^{-i(\mathbf{k}+\mathbf{q})\mathbf{x}\hbar^{-1}}-\hat{a}_\mathbf{k}\hat{a}_\mathbf{q}^\dagger v_\mathbf{k}v^*_\mathbf{q}e^{-i(\mathbf{k}-\mathbf{q})\mathbf{x}\hbar^{-1}}-\right.\\
		-\left.\hat{a}_\mathbf{k}^\dagger\hat{a}_\mathbf{q}v_\mathbf{k}^*v_\mathbf{q}e^{i(\mathbf{k}-\mathbf{q})\mathbf{x}\hbar^{-1}}+\hat{a}^\dagger_\mathbf{k}\hat{a}^\dagger_\mathbf{q}v^*_\mathbf{k}v^*_\mathbf{q}e^{i(\mathbf{k}+\mathbf{q})\mathbf{x}\hbar^{-1}}\right]
	\end{multline}
	\begin{multline}
		\hat{\rchi}^2=\int\frac{\td^3\mathbf{k}\td^3\mathbf{q}}{(2\pi\hbar)^6}\left[\hat{a}_\mathbf{k}\hat{a}_\mathbf{q}v_\mathbf{k}v_\mathbf{q}e^{-i(\mathbf{k}+\mathbf{q})\mathbf{x}\hbar^{-1}}+\hat{a}_\mathbf{k}\hat{a}_\mathbf{q}^\dagger v_\mathbf{k}v^*_\mathbf{q}e^{-i(\mathbf{k}-\mathbf{q})\mathbf{x}\hbar^{-1}}+\right.\\
		+\left.\hat{a}_\mathbf{k}^\dagger\hat{a}_\mathbf{q}v_\mathbf{k}^*v_\mathbf{q}e^{i(\mathbf{k}-\mathbf{q})\mathbf{x}\hbar^{-1}}+\hat{a}^\dagger_\mathbf{k}\hat{a}^\dagger_\mathbf{q}v^*_\mathbf{k}v^*_\mathbf{q}e^{i(\mathbf{k}+\mathbf{q})\mathbf{x}\hbar^{-1}}\right]
	\end{multline}
	\begin{multline}
		\hat{\mathcal{H}}=\frac{c}{2}\int\frac{\td^3\mathbf{k}\td^3\mathbf{q}}{(2\pi\hbar)^3}\left\{\hat{a}_\mathbf{k}\hat{a}_\mathbf{q}\left[\frac{1}{c^2}v'_\mathbf{k}v'_\mathbf{q}-\left(\frac{1}{\hbar^2}\mathbf{k}\mathbf{q}-\mu^2_{\text{eff}}\right)v_\mathbf{k}v_\mathbf{q}\right]\delta^3(\mathbf{k}+\mathbf{q})+\right.\\
		+\left.\hat{a}_\mathbf{k}\hat{a}^\dagger_\mathbf{q}\left[\frac{1}{c^2}v'_\mathbf{k}v^{'*}_\mathbf{q}+\left(\frac{1}{\hbar^2}\mathbf{k}\mathbf{q}+\mu^2_{\text{eff}}\right)v_\mathbf{k}v^*_\mathbf{q}\right]\delta^3(\mathbf{k}-\mathbf{q})+\right.\\
		+\left.\hat{a}^\dagger_\mathbf{k}\hat{a}_\mathbf{q}\left[\frac{1}{c^2}v^{'*}_\mathbf{k}v^{'}_\mathbf{q}+\left(\frac{1}{\hbar^2}\mathbf{k}\mathbf{q}+\mu^2_{\text{eff}}\right)v^*_\mathbf{k}v_\mathbf{q}\right]\delta^3(\mathbf{k}-\mathbf{q})+\right.\\
		+\left.\hat{a}^\dagger_\mathbf{k}\hat{a}^\dagger_\mathbf{q}\left[\frac{1}{c^2}v^{'*}_\mathbf{k}v^{'*}_\mathbf{q}-\left(\frac{1}{\hbar^2}\mathbf{k}\mathbf{q}-\mu^2_{\text{eff}}\right)v^*_\mathbf{k}v^*_\mathbf{q}\right]\delta^3(\mathbf{k}+\mathbf{q})\right\}
	\end{multline}
	\begin{multline}
		\hat{\mathcal{H}}=\frac{c}{2}\int\frac{\td^3\mathbf{k}}{(2\pi\hbar)^3}\left\{\hat{a}_\mathbf{k}\hat{a}_{-\mathbf{k}}\left[\frac{1}{c^2}v'_\mathbf{k}v'_\mathbf{k}+\frac{1}{\hbar^2}\omega^2_\mathbf{k}(t)v_\mathbf{k}v_\mathbf{k}\right]+\right.\\
		+\left.\hat{a}_\mathbf{k}\hat{a}^\dagger_\mathbf{k}\left[\frac{1}{c^2}v'_\mathbf{k}v^{'*}_\mathbf{k}+\frac{1}{\hbar^2}\omega^2_\mathbf{k}(t)v_\mathbf{k}v^*_\mathbf{k}\right]+\right.\\
		+\left.\hat{a}^\dagger_\mathbf{k}\hat{a}_\mathbf{k}\left[\frac{1}{c^2}v^{'*}_\mathbf{k}v^{'}_\mathbf{k}+\frac{1}{\hbar^2}\omega^2_\mathbf{k}(t)v^*_\mathbf{k}v_\mathbf{k}\right]+\right.\\
		+\left.\hat{a}^\dagger_\mathbf{k}\hat{a}^\dagger_{-\mathbf{k}}\left[\frac{1}{c^2}v^{'*}_\mathbf{k}v^{'*}_\mathbf{k}+\frac{1}{\hbar^2}\omega^2_\mathbf{k}(t)v^*_\mathbf{k}v^*_\mathbf{k}\right]\right\}
	\end{multline}
\end{comment}
Substitution of the general expression of $\hat{\rchi}$ (from equation \ref{eq: Expanding General Solution}) and $\hat{\Pi}$ (remembering that $\Pi\equiv\partial_0\rchi$) one will get the following expansion,
\begin{equation}\label{eq: Expanding Hamiltonian}
	\hat{\mathcal{H}}(\eta)=\frac{c}{2}\int\frac{\td^3\mathbf{k}}{(2\pi\hbar)^3}\left[\hat{a}_\mathbf{k}\hat{a}_{-\mathbf{k}}F_\mathbf{k}+\hat{a}^\dagger_\mathbf{k}\hat{a}^\dagger_{-\mathbf{k}}\bar F_\mathbf{k}+\left(2\hat{a}_\mathbf{k}^\dagger\hat{a}_\mathbf{k}+\frac{(2\pi\hbar)^3\hbar c}{2\text{Im}(v\bar v')}\delta^3(\mathbf{0})\right)E_\mathbf{k}\right],
\end{equation}
where the functions $F_\mathbf{k}(t)$ and $E_\mathbf{k}(t)$ are defined as
\begin{subequations}
	\begin{gather}
		F_\mathbf{k}(\eta)=\left(\frac{1}{\hbar c}\right)^2\left[\hbar^2v^{'2}_\mathbf{k}+\omega^2_\mathbf{k}(t)\,c^2 v_\mathbf{k}^2\right],\quad E_\mathbf{k}(\eta)=\left(\frac{1}{\hbar c}\right)^2\left[\hbar^2\big|v'_\mathbf{k}\big|^2+\omega^2_\mathbf{k}(t)\,c^2 \big|v_\mathbf{k}\big|^2\right].\tag{\theequation \,\,a,b}
	\end{gather}
\end{subequations}
\section{Instantaneous Vacuum State}
Note that the only way a vacuum state $|0\rangle$ could remain an eigenstate of the Hamiltonian \ref{eq: Expanding Hamiltonian} at all times, would be if $F_\mathbf{k}(\eta)=0$, at all times, i.e.
\begin{equation}
	F_\mathbf{k}(\eta)=\left(\frac{1}{\hbar c}\right)^2\left[\hbar^2v^{'2}_\mathbf{k}+\omega^2_\mathbf{k}(\eta)\,c^2 v_\mathbf{k}^2\right]=0,
\end{equation}
solving for $v_\mathbf{k}$ gives the following expression
\begin{equation}
	v_\mathbf{k}(\eta)=\text{C}\exp{\left[\pm \frac{c}{i\hbar}\int\omega_\mathbf{k}\left(\eta\right)\td \eta\right]},
\end{equation}
which is not compatible with \ref{eq: Expanding diff v} except for a time independent dispersion relation $\omega_\mathbf{k}$.

The last result implies that, at different times, one can (and should) define different vacuum states; and thus, we will define the \textit{instantaneous vacuum state} $|_{(\eta_0)}0\rangle$ as the one that at some time $t_0$ will minimize the energy density. Since all possible states are related by Bogolyubov transformations, finding the instantaneous vacumm state is the same as finding the set of functions $v_\mathbf{f}$ that are simultaneously solution of \ref{eq: Expanding diff v} and minimize
\begin{equation}
	\langle_{(\eta_0)}0|\hat{\mathcal{H}}(\eta_0)|_{(\eta_0)}0\rangle=\rho(\eta_0)\delta^3(\mathbf{0})=\frac{\hbar c^2 \,\delta^3(\mathbf{0})}{4\text{Im}(v\bar v')}\int\td^3\mathbf{k}\,E_\mathbf{k}
\end{equation}
To minimise the energy density of the vacuum state is to find the set of functions $v_\mathbf{k}$ that minimise $E_\mathbf{k}$. Suppose that $v_\mathbf{k}$ can be written as
\begin{equation}
	v_\mathbf{k}=r_\mathbf{k}e^{i\alpha_\mathbf{k}}
\end{equation}
since Im$(v\bar v')$ was constant through time
\begin{equation}
	r_\mathbf{k}^2\alpha'_\mathbf{k}=-\text{Im}(v\bar v')
\end{equation}
this means
\begin{equation}
	E_\mathbf{k}=\left(\frac{1}{\hbar c}\right)^2\left\{\hbar^2\left[r^{'2}_\mathbf{k}+\text{Im}^2\left(v\bar v'\right)\frac{1}{r_\mathbf{k}^2}\right]+\omega^2_\mathbf{k}\,c^2r_\mathbf{k}^2\right\}
\end{equation}
the minimum of this function must fulfil $r'_\mathbf{k}(\eta_0)=0$. Now, if $\omega_\mathbf{k}^2(\eta_0)$ and $\text{Im}(v\bar v')$ have the same sign, the minimum of $E_\mathbf{k}$ happens when $r_\mathbf{k}(\eta_0)=\left[\frac{\hbar\text{Im}(v\bar v')}{\omega_\mathbf{k}(\eta_0)\,c}\right]^{1/2}$.

If there is a minimum, then
\begin{equation}
	v_\mathbf{k}(\eta_0)=\left[\frac{\hbar\text{Im}(v\bar v')}{\omega_\mathbf{k}(\eta_0)\,c}\right]^{1/2}\!\!\!\!e^{i\alpha_\mathbf{k}(\eta_0)}\hspace{1.0cm}v'_\mathbf{k}(\eta_0)=-c\frac{\omega_\mathbf{k}(\eta_0)}{ih}\,v_\mathbf{k}(\eta_0)
\end{equation}
under these functions,
\begin{equation}
	E_\mathbf{k}(\eta_0)=2\frac{\text{Im}(v\bar v')}{\hbar c}\omega_\mathbf{k}(\eta_0)\hspace{1.0cm}F_\mathbf{k}(\eta_0)=0
\end{equation}
meaning
\begin{equation}
	\hat{\mathcal{H}}(\eta_0)=\text{Im}(v\bar v')\frac{1}{\hbar}\int\frac{\td^3\mathbf{k}}{(2\pi\hbar)^3}\left(2\hat{a}_\mathbf{k}^\dagger\hat{a}_\mathbf{k}+\frac{(2\pi\hbar)^3\hbar c}{2\text{Im}(v\bar v')}\delta^3(\mathbf{0})\right)\omega_\mathbf{k}(t_0)
\end{equation}
which is equivalent to the standard Hamiltonian for a scalar field without the presence of gravity.

\vspace*{0.5cm}

But what is the energy of the instantaneous vacuum at a different time? The relation of such energies can be computed considering that the field at some time $\eta$ can be described as some Bogoliubov transformation of the same field at a time $\eta_0$.

From equations \ref{eq: Bogoliubov Coefficients General} one gets that the Bogoliubov coefficients will be given by
\begin{subequations}
	\begin{gather}
		\alpha_{\mathbf{kp}}=\frac{(2\pi\hbar)^3\hbar c}{2\text{Im}(v\bar v')}\langle \rchi_\mathbf{k}(\eta_0),\,\rchi_\mathbf{p}(\eta)\rangle,\hspace{1.0cm}\beta_{\mathbf{kp}}=-\frac{(2\pi\hbar)^3\hbar c}{2\text{Im}(v\bar v')}\langle \rchi_\mathbf{k}(\eta_0),\,\bar \rchi_\mathbf{p}(\eta)\rangle,\tag{\theequation \,\,a,b}
	\end{gather}
\end{subequations}
and, since the field can be written as $\rchi_\mathbf{k}=v_\mathbf{k}e^{i\mathbf{kx}\sfrac{}{h}}$; from the definition of the binary operation $\langle\cdot,\cdot\rangle$ in expression \ref{eq: General Scalar Inner Product}, one can see that
\begin{subequations}
	\begin{gather}
		\alpha_{\mathbf{kp}}\propto \delta^3(\mathbf{k}-\mathbf{p}),\hspace{1.0cm}\beta_{\mathbf{kp}}\propto \delta^3(\mathbf{k}+\mathbf{p}),\tag{\theequation \,\,a,b}
	\end{gather}
\end{subequations}
and thus, it is possible to write $v_\mathbf{k}$ at an arbitrary time $t$ as
\begin{equation}
	v_\mathbf{k}(\eta)=\alpha_\mathbf{k}v_\mathbf{k}(\eta_0)+\beta_\mathbf{k}\bar v_\mathbf{k}(\eta_0);
\end{equation}
where, recalling that $\text{Im}(v\bar v')$ is constant through time, the relation between $\alpha_\mathbf{k}$ and $\beta_\mathbf{k}$ must be
\begin{equation}\label{eq: Bogoliubov coefficients relation Expanding}
	|\alpha_\mathbf{k}|^2-|\beta_\mathbf{k}|^2=1.
\end{equation}
The energy of the instantaneous vacuum state $|_{(\eta_0)}0\rangle$ at a time $\eta$ is given by $\langle_{(\eta_0)}0|\hat{\mathcal{H}}(\eta)|_{(\eta_0)}0\rangle$, were the hamiltonian $\hat{\mathcal{H}}(\eta)$ is given by expression \ref{eq: Expanding Hamiltonian}; to compute that, lets first obtain


\begin{equation}
	\langle_{(\eta_0)}0|\hat{a}^\dagger_\mathbf{k}(\eta)\hat{a}_\mathbf{k}(\eta)|_{(\eta_0)}0\rangle=\Big|\beta_\mathbf{k}\Big|^2\frac{(2\pi\hbar)^3\hbar c}{2\text{Im}(v\bar v')}\delta^3(\mathbf{0})
\end{equation}
therefore the energy of the instantaneous vacuum state $|_{(\eta_0)}0\rangle$ at a time $\eta$ is given by
\begin{equation}
	\langle_{(\eta_0)}0|\hat{\mathcal{H}}(\eta)|_{(\eta_0)}0\rangle=\delta^3(\mathbf{0})\int\td^3\mathbf{k}\left(\frac{1}{2}+\Big|\beta_\mathbf{k}\Big|^2\right)c\,\omega_\mathbf{k}(\eta)\geq \langle_{(\eta_0)}0|\hat{\mathcal{H}}(\eta_0)|_{(\eta_0)}0\rangle.
\end{equation}
As expected, the energy density will be bigger at a different time $\eta$ if $\beta_\mathbf{k}\not=0$ for all $\mathbf{k}$, since the definition of $|_{(\eta_0)}0\rangle$ is such state that minimizes the energy density at some particular time $\eta_0$; meaning that at a different time, other state must have a lower energy density.
\section{Case of Study: de Sitter Universe}
The de Sitter Universe is a flat FLRW metric with no matter or radiation, but it does have a a positive cosmological constant $\Lambda$. Per the Friedmann equations,
\begin{equation}
	\left(\frac{\dot{a}}{a}\right)^2=\frac{8\pi G\rho+\Lambda c^2}{3}-\frac{\kappa c^2}{a^2},\quad \left(\rho=\kappa=0\right)
\end{equation}
the expansion parameter $a(t)$ will be equal to
\begin{equation}
	a(t)=a_1e^{H_\Lambda t}+a_2e^{-H_\Lambda t}\;,
\end{equation}
where $H_\Lambda=\sqrt{\sfrac{\Lambda c^2}{3}}$ it's the Hubble-Lemaître constant. The most common choice is to set $a_2=0$, and thus, consider an always expanding universe.

The line element describing the motion of particles through this universe is given by
\begin{equation}
	\td l^2=c^2\td t^2-a^2(t)\td\mathbf{x}^2,
\end{equation}
more commonly expressed as a function of the conformal time $\eta$ defined as
\begin{equation}
	\eta\equiv -\int_t^\infty\frac{\td t'}{a(t')}=-\frac{1}{a_1\,H_\Lambda}e^{-H_\Lambda t}=-\frac{1}{a(t)H_\Lambda};
\end{equation}
and thus, the line element will be given by
\begin{equation}
	\td l^2= \frac{1}{H_\Lambda\eta^2}\left[c^2\td\eta^2-\td \mathbf{x}^2\right],
\end{equation}
which has the same form as \ref{eq: Conformal FLRW}. 

Now, from equation \ref{eq: FLRW Curvature} one can compute the curvature scalar $R=H_\Lambda^2\sfrac{12}{c^2}$, and thus, the dispersion relation \ref{eq: Expanding dispersion relation} will be given by the following expression
\begin{equation}\label{eq: de Sitter dispersion relation}
	\omega^2_\mathbf{k}(\eta)=\mathbf{k}^2+\left[\left(\frac{mc^2}{H_\Lambda}\right)^2+2\left(6\xi-1\right)\hbar^2\right]\frac{1}{c^2\eta^2},
\end{equation}
from which one might obtain the solutions of the differential equation \ref{eq: Expanding diff v}; to do so it is best to use the following change of variables,
\begin{subequations}
	\begin{gather}
		s\equiv -k\eta\frac{c}{\hbar},\hspace{1.5cm}v_\mathbf{k}\equiv \sqrt{s}f(s),\tag{\theequation \,\,a,b}
	\end{gather}
\end{subequations}
obtaining the Bessel's differential equation
\begin{equation}
	s^2\frac{\td^2f}{\td s^2}+s\frac{\td f}{\td s}+\left(s^2-\nu^2\right)f(s)=0,
\end{equation}
with a parameter
\begin{equation}
	\nu^2\equiv \left(3-16\xi\right)\frac{3}{4}-\left(\frac{mc^2}{H_\Lambda\hbar }\right)^2.
\end{equation}
The solutions of such differential equation are given by the so called Bessel functions of the first kind $J_\nu(s)$ and $Y_\nu(s)$; therefore the $v_\mathbf{k}$ functions can be deduced as
\begin{equation}\label{eq: de Sitter general v}
	f(s)=AJ_\nu(s)+BY_\nu(s)\implies v_\mathbf{k}(\eta)=\sqrt{k|\eta|\frac{c}{\hbar}}\left[A_\mathbf{k}J_\nu(k|\eta|\frac{c}{\hbar})+B_\mathbf{k}Y_\nu(k|\eta|\frac{c}{\hbar})\right]
\end{equation}
For $\nu^2\geq 0$, both $J_\nu$ and $Y_\nu$ will be real functions, but for $\nu<0$ they will be complex functions \cite{BesselComplex}; for simplicity, we will focus on the $\nu\geq 0$ case. In addition, consider that the choice of $\text{Im}(v\bar v')$ will translate in a restriction on the relation between the  $A_\mathbf{k},\,B_\mathbf{k}$ parameters;
\begin{equation}
	\text{Im}\left(v\bar{v}'\right)=ik^2|\eta|\frac{c}{\hbar}\left(A_\mathbf{k}\bar{B}_\mathbf{k}-\bar{A}_\mathbf{k}B_\mathbf{k}\right)W\left[J_\nu\left(k|\eta|\right),\,Y_\nu\left(k|\eta|\right)\right],
\end{equation}
where $W[f,g]\equiv fg'-f'g$ its the Wronskian, particularly \cite[\href{https://dlmf.nist.gov/10.5}{10.5.2}]{DLMF} for the Bessel functions, one gets $W\left[J_\nu\left(s\right),\,Y_\nu\left(s\right)\right]=\frac{2}{\pi s}$
and thus, the relation between the  $A_\mathbf{k},\,B_\mathbf{k}$ parameters will be
\begin{equation}\label{eq: Relation A B de Sitter}
	\left(A_\mathbf{k}\bar{B}_\mathbf{k}-\bar{A}_\mathbf{k}B_\mathbf{k}\right)=-i\frac{\pi}{2k}\text{Im}\left(v\bar{v}'\right).
\end{equation}
As previously seen, the choice of a set of solutions of the Klein-Gordon equation, defines the choice of a vacuum; an example is the instantaneous vacuum, but for a de Sitter background there is a special choice, the Bunch-Davies vacuum\footnote{Also known as euclidean vacuum.}.
\subsection{Bunch-Davies Vacuum}
Consider the behaviour of the field on the distant past, that is, $k|\eta|\to\infty$; at this moment, the dispersion relation \ref{eq: de Sitter dispersion relation} is given by $\omega_\mathbf{k}\approx k$, therefore, the solutions of \ref{eq: Expanding diff v} are given by
\begin{equation}
	v_\mathbf{k}(\eta)\approx \frac{1}{\sqrt{2\omega_\mathbf{k}}}e^{i\omega_\mathbf{k}\eta},
\end{equation}
which is the usual flat temporal mode function. The same limit can be taken over the solution \ref{eq: de Sitter general v}, 
considering the asymptotic expressions for the Bessel functions \cite[\href{https://dlmf.nist.gov/10.7}{10.7.8}]{DLMF}, one obtains
\begin{subequations}
	\begin{gather}
		v_\mathbf{k}(\eta)\approx \sqrt{\frac{2}{\pi}}\left[A_\mathbf{k}\cos\lambda_\nu+B_\mathbf{k}\sin\lambda_\nu\right],\quad \lambda_\nu\equiv k|\eta|\frac{c}{\hbar}-\frac{\pi}{2}\nu-\frac{\pi}{4}.\tag{\theequation \,\,a,b}
	\end{gather}
\end{subequations}
One could easily recover the flat temporal mode function by imposing the relation 
$B_\mathbf{k}=iA_\mathbf{k}$; this choice, alongside the relation \ref{eq: Relation A B de Sitter}, meaning that  $|A_\mathbf{k}|=\sqrt{\sfrac{\pi}{4 k}\text{Im}\left(v\bar{v}'\right)}$; and therefore, the Bunch-Davies temporal mode functions are given (except for an irrelevant phase) by
\begin{equation}
	v_\mathbf{k}(\eta)=\sqrt{\frac{\pi}{4 }|\eta|\text{Im}\left(v\bar{v}'\right)}H^{(1)}_\nu\left(k|\eta|\right),
\end{equation}
where $H^{(1)}_\nu(s)\equiv J_\nu(s)+iY_\nu(s)$ is known as the Hankel function of the first kind.

This vacuum state is of particular interest since it is not time dependent, is invariant under the de Sitter symmetry group and, for a conformal theory (i.e. $m=0$ and $\xi=\sfrac{1}{6}$) the temporal mode function is given by\footnote{For such theory, $\omega_\mathbf{k}=k$ and $\nu=\sfrac{1}{2}$; and for such value, the Hankel function can be written as $$H^{(1)}_{\sfrac{1}{2}}(s)=-i\sqrt{\frac{2}{\pi s}}e^{is}.$$}
\begin{equation}
	v_\mathbf{k}(\eta)=-i\sqrt{\frac{\hbar\text{Im}\left( v\bar{v}'\right)}{ 2\omega_\mathbf{k}c}}e^{i\omega_\mathbf{k}|\eta|c\sfrac{}{\hbar}}
\end{equation}
which is (up to an irrelevant phase) the standard plane wave temporal mode function.

Any other vacuum state can be obtained by a Bogoliubov transformation of the Bunch-Davies vacuum; from the relation of the Bogoliubov parameters in expression \ref{eq: Bogoliubov coefficients relation Expanding}, one could define a new set of temporal mode functions $\left\{u_\mathbf{k}\right\}$ as
\begin{equation}
	u_\mathbf{k}(\eta)\equiv \cosh\alpha \,v_\mathbf{k}(\eta)+e^{i\beta}\sinh\alpha \,\bar{v}_\mathbf{k}(\eta),
\end{equation}
where $0\leq \alpha<\infty$ and $-\pi\leq\beta<\pi$ are constant parameters defining the transformation. This new vacuum state denoted by $|_{(\alpha,\,\beta)}0\rangle$ is known as Mottola-Allen vacuum.
