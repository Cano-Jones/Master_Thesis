One of the most important objects in a theory of quantum fields are the correlation functions; in particular one of them takes on greater significance: the Feynman propagator $G_F(x,x')$, which can be defined for a free scalar field as the solution of the following homegeneous equation
\begin{equation}
	\left[\partial_\nu\partial^\nu+\mu^2+\xi R(x)\right]G_F(x,x')=-\left[-g(x)\right]^{-\sfrac{1}{2}}\delta^{(d+1)}(x-x').
\end{equation}

In what follows, we will present a particularly useful representation of the scalar field propagator following the treatment given by \cite{DeWitt}. This representation has important applications on regularization, as was presented in the corresponding section of this thesis (sec. 4). 

Before presenting the mathematical formalism, one must introduce the Riemann normal coordinates $y^\mu$ for a given given point $x$, whose origin is considered $x'$. Using these particular coordinates, one might expand
\begin{equation}\label{eq: Metric in Riemann coordinates}
	g_{\mu\nu}(x)=\eta_{\mu\nu}+\frac{1}{3}R_{\mu\alpha\nu\beta}\,y^\alpha y^\beta-\frac{1}{6}\nabla\gamma R_{\mu\alpha\nu\beta}\,y^\alpha y^\beta y^\gamma+\left[\frac{1}{20}\nabla_\gamma\nabla_\delta\; R_{\mu\alpha\nu\beta}+\frac{2}{45}R_{\alpha\mu\beta\lambda}R^\lambda_{\;\gamma\nu\delta}\right]y^\alpha y^\beta y^\gamma y^\delta+\hdots,
\end{equation}
where all coefficients are evaluated at $y=0$.

Next, a new two-point function is defined from $G_F(x,x')$ as
\begin{equation}
	\mathcal{G}_F(x,x')\equiv \left[-g(x)\right]^{-\sfrac{1}{4}}G_F(x,x'),
\end{equation}
alongside its Fourier transform
\begin{equation}
	\mathcal{G}_F(x,x')=\int \frac{\td^{d+1}k}{(2\pi\hbar)^{d+1}}\mathcal{G}_F(k)e^{-iky\hbar^{-1}};
\end{equation}
where $ky\equiv \eta_\mu\nu k^\mu y^\nu$. 

By using Riemann normal coordinates in the propagator equation, and converting to $k$-space, $\mathcal{G}_F(k)$ can be solved by iteration to any desired derivative order (also known as adiabatic order). In particular, for order $4$ one obtains
\begin{multline}
	\mathcal{G}_F(k)\approx\hbar^2\left(k^2-\mu^2\hbar^2\right)^{-1}-\hbar^4\left(\frac{1}{6}-\xi\right)\left[R-\frac{i}{2}\left(\nabla_\alpha R\right)\partial^\alpha+\frac{1}{3}a_{\alpha\beta}\partial^\alpha\partial^\beta\right]\left(k^2-\mu^2\hbar^2\right)^{-2}+\\
	+\hbar^3\left[\frac{2}{3}a^\lambda_\lambda-\left(\frac{1}{6}-\xi\right)^2R^2\right]\left(k^2-\mu^2\hbar^2\right)^{-3},
\end{multline}
with the derivative referring to $k$-space, that is, $\partial_\alpha\equiv\sfrac{\partial }{\partial k^\alpha}$; and the element $a_{\alpha\beta}$ defined as
\begin{equation}
	a_{\alpha\beta}\equiv \left[\frac{1}{2}\left(\frac{1}{6}-\xi\right)+\frac{1}{120}\right]\nabla_\alpha\nabla_\beta\, R-\frac{1}{40}\nabla_\lambda\nabla^\lambda\,R_{\alpha\beta}-\frac{1}{30}R^{\;\lambda}_\alpha R_{\lambda\beta}+\frac{1}{60}R^{\kappa\;\lambda}_{\;\alpha\;\beta}R_{\kappa\lambda}+\frac{1}{60}R^{\lambda\mu\kappa}_{\;\;\;\alpha}R_{\lambda\mu\kappa\beta}.
\end{equation}

If one were to write
\begin{equation}\label{eq: g_F before integral}
	\mathcal{G}_F(x,x')=\int \frac{\td^{d+1}k}{(2\pi\hbar)^{d+1}}e^{iky\hbar^{-1}}\left[a_0(x,x')-a_1(x,x')\frac{\partial}{\partial \mu^2}+a_2(x,x')\left(\frac{\partial}{\partial \mu^2}\right)^2\right]\hbar^2\left(k^2-\mu^2\hbar^2\right)^{-1},
\end{equation}
then (up to fourth adiabatic order), the element $a_i(x,x')$  must equal
\begin{subequations}
	\begin{gather}
		a_0(x,x')=1,\quad a_1(x,x')=\left(\frac{1}{6}-\xi\right)\left(R-\frac{1}{2}\nabla_\alpha R\,y^\alpha\right)-\frac{1}{3}a_{\alpha\beta} y^\alpha y^\beta\tag{\theequation \,\,a,b}\\
		a_2(x,x')=\frac{1}{2}\left(\frac{1}{6}-\xi\right)^2R^2+\frac{1}{3}a^\lambda_{\;\lambda}\tag{\theequation \,\,c};
	\end{gather}
\end{subequations}
all right hand side elements evaluated at $x'$.

The next step is to consider the integral representation
\begin{equation}
	\hbar^2(k^2-\mu^2\hbar^2+i\epsilon)^{-1}=-\int_0^\infty e^{-is(k^2-\mu^2\hbar^2+i\epsilon)\hbar^{-2}}\td(is),
\end{equation}
which, when added into equation \ref{eq: g_F before integral} will give as a result
\begin{equation}
	\mathcal{G}_F(x,x')=-(4\pi)^{-\sfrac{(d+1)}{2}}\int_0^\infty F(x,x';is)\exp{\left[-is\mu^2+\frac{\sigma(x,x')}{2is}\right]}\,\td(is),
\end{equation}
where $\sigma(x,x')\equiv\sfrac{1}{2}y_\alpha y^\alpha$ can be be understood as half of the squared geodesic distance between $x$ and $x'$. Here we have introduced the function $F(x,x';is)$ representing the dependence of the $a_i(x,x')$ terms up to an adiabatic order $N$, such that
\begin{equation}
	F(x,x';is)\equiv \sum_{n=0}^{\sfrac{N}{2}}a_n(x,x')(is)^n.
\end{equation}

Finally, using the relation between $\mathcal{G}_F(x,x')$ and the propagator $G_F(x,x')$, one can obtain what is known as the DeWitt-Schinger representation of the scalar propagator; given by
\begin{equation}
	G_F^{\text{DS}}(x,x')\equiv -i\frac{\Delta^{1/2}(x,x')}{(4\pi)^{\sfrac{(d+1)}{2}}}\int^\infty_0F(x,y;is)\exp{\left[-is\mu^2+\frac{\sigma(x,x')}{2is}\right]}(is)^{-\sfrac{(d+1)}{2}}\,\td (is);
\end{equation}
where $\Delta(x,x')$ is the so called Van Vleck determinant, defined as 
\begin{equation}
	\Delta(x,x')\equiv -\text{det}\left[\partial_\mu\partial_\nu\sigma(x,x')\right]\left[g(x)g(x')\right]^{-\sfrac{1}{2}}.
\end{equation}

As a last note, the equality only really holds if the closed expression of the function $F(x,x';is)$ is known, otherwise it must be understood as a limiting expansion, in which each $a_i(x,x')$ term follows some recursive rules.