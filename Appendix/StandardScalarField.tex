Throughout this thesis, the main field used to introduce the theory of quantum field in curved spacetimes, was the scalar field. Therefore, it would be good practice to introduce the theory of scalar fields on a Minkowski background.
\vspace{0.25cm}
\section*{Introduction}
The simplest action for a (real) scalar field without interactions, might be 
\begin{equation}\label{eq: Minkowski scalar action}
	S[\phi]=\int\frac{1}{2}\Big[\partial_\nu\phi\,\partial^\nu\phi-\mu^2\phi^2\Big]\;\td^4 x.
\end{equation}
From this action, one can obtain the equations of motion of the field from the Euler-Lagrange equations, which would yield as a result the so-called Klein-Gordon equation,
\begin{equation}
	\left(\partial_\nu\partial^\nu-\mu^2\right)\phi=0,
\end{equation}
with solutions of the form
\begin{equation}\label{eq: Klein-Gordon mode solution}
	\phi_\mathbf{k}=a_\mathbf{k}e^{ikx\,\hbar^{-1}}+\bar a_\mathbf{k}e^{-ikx\,\hbar^{-1}}.
\end{equation}
Substitution of this solution in the Klein-Gordon equation results in the following dispersion relation
\begin{equation}
	k_\nu k^\nu=\hbar^2\mu^2,
\end{equation}
which gives a relation between the parameter $\mu$ and the mass of the field $m$ through $\mu=\sfrac{mc}{\hbar}$.

The most general solution of the Klein-Gordon equation can be written as a mode expansion of solutions of the form \ref{eq: Klein-Gordon mode solution}, which would be
\begin{equation}
	\phi(x)=\int\frac{\td^3\mathbf{k}}{(2\pi\hbar)^32k_0}\phi_{\mathbf{k}}=\int\frac{\td^3\mathbf{k}}{(2\pi\hbar)^32k_0}\left(a_\mathbf{k}e^{ikx\,\hbar^{-1}}+\bar a_\mathbf{k}e^{-ikx\,\hbar^{-1}}\right);
\end{equation}
where $\frac{\td^3\mathbf{k}}{(2\pi\hbar)^32k_0}$ is a Lorentz-invariant measure. It is convenient to redefine $a_\mathbf{k}\to a_\mathbf{k}(2k_0)^{-\sfrac{1}{2}}$, and thus, the field would be described as
\begin{equation}
	\phi(x)=\int\frac{\td^3\mathbf{k}}{(2\pi\hbar)^3\sqrt{2k_0}}\left(a_\mathbf{k}e^{ikx\,\hbar^{-1}}+\bar a_\mathbf{k}e^{-ikx\,\hbar^{-1}}\right).
\end{equation}
\section*{Quantization}
The methodology to canonically quantize a field comes from the promotion of the field $\phi(x)$ and its conjugated momenta $\Pi(x)\equiv\partial_{ct}$ into quantum operators,
\begin{equation*}
	\phi(x)\longrightarrow\hat{\phi}(x),\hspace{1.0cm}\Pi(x)\longrightarrow\hat{\Pi}(x),
\end{equation*}
to do so, the most common procedure is to promote the mode constant factors to quantum operators, which will be known as annihilation and creator operators, 
\begin{equation*}
	a_\mathbf{k}\longrightarrow\hat{a}_\mathbf{k},\hspace{1.0cm}\bar a_\mathbf{k}\longrightarrow\hat{a}_\mathbf{k}^\dagger.
\end{equation*}
This will imply that the quantum field operator $\hat{\phi}(x)$ will have the form
\begin{equation}
	\hat\phi(x)=\int\frac{\td^3\mathbf{k}}{(2\pi\hbar)^3\sqrt{2k_0}}\left(\hat a_\mathbf{k}e^{ikx\,\hbar^{-1}}+ \hat a_\mathbf{k}^\dagger e^{-ikx\,\hbar^{-1}}\right).
\end{equation}
In addition to the promotion, some commutation rules must be imposed, to do so, Dirac proposes the following procedure: replacing the Poisson Brackets of the phase space with commutators, as
\begin{equation}
	\left\{A,B\right\}\to\frac{1}{i\hbar}\left[\hat{A},\hat{B}\right];
\end{equation}
the Poisson Bracket for a coordinate $q_i$ and a conjugate momentum $p_j$ is $\left\{q_i,p_j\right\}=\delta_{ij}$, therefore it is natural to consider as commutation rules the following,


\begin{subequations}
	\begin{gather}
		\left[\hat{\phi}(\mathbf{x}),\,\hat{\Pi}(\mathbf{y})\right]=i\hbar \,\delta^3\left(\mathbf{x}-\mathbf{y}\right)\hspace{1.0cm}\left[\hat{\phi}(\mathbf{x}),\,\hat{\phi}(\mathbf{y})\right]=\left[\hat{\Pi}(\mathbf{x}),\,\hat{\Pi}(\mathbf{y})\right]=0.\tag{\theequation \,\,a-c}.
	\end{gather}
\end{subequations}
Substitution of the quantum field expression in these commutators will give the needed commutation rules for the annihilation and creation operators:
\begin{subequations}\label{eq: general a operator commutator rules}
	\begin{gather}
		\left[\hat{a}_\mathbf{k},\,\hat{a}_\mathbf{q}^\dagger\right]=\left(2\pi\hbar\right)^3\hbar^2 \delta^3\left(\mathbf{k}-\mathbf{q}\right),\hspace{1.0cm}\left[\hat{a}_\mathbf{k},\,\hat{a}_\mathbf{q}\right]=\left[\hat{a}_\mathbf{k}^\dagger,\,\hat{a}_\mathbf{q}^\dagger\right]=0.\tag{\theequation \,\,a-c}
	\end{gather}
\end{subequations}
\subsection*{Hamiltonian and Fock space}
The most relevant quantum operator is without doubt, the Hamiltonian $\hat{\mathcal{H}}$, which is given by Noether's theorem as the conserved current
\begin{equation}
	\hat{\mathcal{H}}=\int\left[\frac{\partial\hat{\mathcal{L}}}{\partial\left(\partial_0\hat{\phi}\right)}-\hat{\mathcal{L}}\right]\td^3\mathbf{x};
\end{equation}
for the given field, one obtains the following expression for the Hamiltonian
\begin{equation}
	\hat{\mathcal{H}}=\int\left(\hat{\Pi}\partial_{0}\hat{\phi}-\hat{\mathcal{L}}\right)\td^3\mathbf{x}=\int\frac{c}{2}\left[\hat{\Pi}^2+\left(\mathbf{\nabla}\hat{\phi}\right)^2+\mu^2\hat{\phi}^2\right]\td^3\mathbf{x};
\end{equation}
and once the quantum field expansion is substituted, it will be simplified as
\begin{equation}
	\hat{\mathcal{H}}=\int E_\mathbf{p}\left[\hat{a}_\mathbf{p}\hat{a}_\mathbf{p}^\dagger+\frac{1}{2}(2\pi\hbar)^3\hbar^2\delta(\mathbf{0})\right]\frac{\td^3\mathbf{p}}{(2\pi\hbar)^3\hbar^2}.
\end{equation}
Note that the constant term
\begin{equation}
	\hat{\mathcal{H}}_{\text{div}}=\frac{1}{2}(2\pi\hbar)^3\hbar^2\delta(\mathbf{0})\int E_\mathbf{p}\frac{\td^3\mathbf{p}}{(2\pi\hbar)^3\hbar^2}
\end{equation}
is divergent, and known as vacuum energy; it is useful to define a `normal" ordered Hamiltonian without this term, which will be zero for a vacuum state
\begin{equation}
	:\!\hat{\mathcal{H}}\!:=\int E_\mathbf{p}\;\hat{a}_\mathbf{p}\hat{a}_\mathbf{p}^\dagger\frac{\td^3\mathbf{p}}{(2\pi\hbar)^3\hbar^2}.
\end{equation}

The Fock space $\{|\mathbf{p}\rangle\}$ generated by the Hamiltonian is formed from a vacuum (no particles) state $|0\rangle$ which is annihilated by $\hat{a}_\mathbf{p}$, i.e.
\begin{equation}
	\hat{a}_\mathbf{p}|0\rangle=0;
\end{equation}
other one particle states, with a given momentum $\mathbf{p}$, are formed from the vacuum state after applying the creator operator
\begin{equation}
	|\mathbf{p}\rangle\equiv \hat{a}^\dagger_{\mathbf{p}}|0\rangle;
\end{equation}
similar to a quantum oscillator. Multiparticle states are formed after the chain use of the creator operators
\begin{equation}
	|\mathbf{p}_1,\mathbf{p}_2,\hdots\rangle\equiv \hdots \hat{a}^\dagger_{\mathbf{p}_2}\hat{a}^\dagger_{\mathbf{p}_1}|0\rangle;
\end{equation}
note that this quantum states are bosonic, since
\begin{equation}
	|\mathbf{p}_1,\mathbf{p}_2\rangle=\hat{a}^\dagger_{\mathbf{p}_2}\hat{a}^\dagger_{\mathbf{p}_1}|0\rangle=\left[\hat{a}^\dagger_{\mathbf{p}_2},\hat{a}^\dagger_{\mathbf{p}_1}\right]|0\rangle+\hat{a}^\dagger_{\mathbf{p}_1}\hat{a}^\dagger_{\mathbf{p}_2}|0\rangle=|\mathbf{p}_2,\mathbf{p}_1\rangle.
\end{equation}