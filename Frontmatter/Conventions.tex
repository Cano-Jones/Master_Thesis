
The chosen convention for the metric signature will be $(+,\,-,\,-,\,-)$ as in \cite{BirrelDavies} and most literature on Particle Physics. Common conventions and nomenclatures in Mathematics and Physics are used throughout the text, some of which are considered to be relevant:\newline

\begin{tabular}{p{1.75cm}p{11cm}}
	
	$x^\mu,\,x$ & four-vector \\
	$\mathbf{x}$ & spacial vector \\
	$g_{\mu\nu}$ & general spacetime metric \\
	$\eta_{\mu\nu}$ & Minkowski spacetime metric \\
	$g$ & determinant of $g_{\mu\nu}$ \\
	$\nabla_\mu$ & covariant derivative \\
	$S[\phi]$ & action functional of a field $\phi$ and its derivatives \\
	$\bar{z}$ & complex conjugate of $z$ \\
	$A^\dagger$ & hermitian conjugate of $A$\\
	$R^\alpha_{\;\beta\gamma\delta}$ & Riemann tensor \quad $\equiv \nabla_\delta\Gamma^\alpha_{\;\beta\gamma}-\nabla_\delta\Gamma^\gamma_{\;\beta\alpha}+\hdots$\\
	$f\,\overset{\leftrightarrow}{\nabla}_\mu\, g$ &$\equiv f\nabla_\nu g-\left(\nabla_\mu f\right)g$ \\
	$\gamma^\mu$ & covariant Gamma matrices, $\left\{\gamma^\mu,\,\gamma^\nu\right\}=2g^{\mu\nu}$ \\
	$G,\,c,\,\hbar$ & standard universal constants, not necessarily in natural units \\
	
\end{tabular}

\vspace*{0.5cm}
Other notation will be introduced as needed.
\begin{comment}
	\vspace*{2.5cm}
	\paragraph{Choice of Units\\}
	Most of the literature on quantum field theory and general relativity are presented with the choice of use of natural units, reason for which it was considered appropriate to add here a reminder of some unit quantities.\newline
	
	
	\begin{tabular}{p{1.75cm}p{7cm}p{1.75cm}}
		$[S]=[\hbar]$ & the action and Planck's constant & M$\cdot$L$^2\cdot$T$^{-1}$\\
		$[\phi(x)]$ & the scalar field & $[\hbar]^{\sfrac{1}{2}}\cdot$L$^{-1}$\\
		$[\Pi(x)]$ & the scalar field conjugate momenta & $[\hbar]^{\sfrac{1}{2}}\cdot$L$^{-2}$\\
		
	\end{tabular}
	\begin{multicols}{2}
		\begin{itemize}
			\item $[S]=[\hbar]$,
			\item $[a]=[\xi]=1$,
			\item $[\mu]=[L]^{-1}$,
			\item $[R]=[L]^{-2}$,
			\item $[\phi]=[\rchi]=[\hbar]^{1/2}[L]^{-1}$,
			\item $[\Pi]=[\hbar]^{1/2}[L]^{-2}$,
			\item $[a_\mathbf{k}]=[\hbar]^{1/2}[L]^2$;
		\end{itemize}
	\end{multicols}
	
	where $[L]$ is to be understood as "length units".
\end{comment}