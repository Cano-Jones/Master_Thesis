\vspace*{1.0cm}
	\paragraph{Abrstract\\}
	\begin{minipage}{12cm}
		
	\vspace*{0.5cm}
	
	Quantum Field Theory is the fundamental theoretical structure of the Standard Model of elementary particles. This theory is formulated in a Minkowski space-time. However, the actual space-time metric is never of that type. On Earth, even at short distances, the metric is affected by both the force of the Earth's gravity and the solar force, but especially by the acceleration of the Earth's motion. At large distances the cosmological data lead one to think that, on average, the metric is of the Friedmann-Lemaitre-Roberson-Walker type.
	
	\vspace*{0.2cm}
	
	The analysis of the quantization of fields in the presence of gravitational fields involves a number of theoretical connotations that we intend to explore in this paper. How Quantum Field Theory can be adapted when considering this gravitational background is the object of the proposal. In particular, how the structure of the quantum vacuum is affected when space-time is neither asymptotically Minkowskian, as is the case in the current Cosmological Model LCDM. There are a number of attempts to address the problem but to date none provides a satisfactory solution; the aim of the present work is to find the one that best fits the experimental observations.
	
	\end{minipage}
	
	\vspace*{\fill}
	
	
	This work, © 2023 by Cano Jones, Alejandro is licensed under CC BY-NC-ND 4.0. To view a copy of this license, visit \href{http://creativecommons.org/licenses/by-nc-nd/4.0/}{http://creativecommons.org/licenses/by-nc-nd/4.0/}
	
	
	\begin{flushright}
		\includegraphics[height=1cm]{Images/Logos/by-nc-nd.eu}
	\end{flushright}
