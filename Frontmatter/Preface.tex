Over the last century, Physics has evolved into two separate fundamental theories that cover two widely different energy ranges. On one side of the spectrum, Quantum Field Theory has been able to model small scale phenomena and the nature of the fundamental forces that bind matter together; and on the other, General Relativity sheds light on the proper nature of gravity and the stage in which particles move. Although these two seemingly different frameworks appear to pass all observational and experimental tests, one can find a plethora of arguments to consider that a more fundamental theory of Quantum Gravity must exist. Such unifying theory has been pursued for decades, with no definitive answer; and thus, we are left with what is expected to be its low energy limit: the theory of Quantum Fields in Curved Spacetimes, which is the focus of this thesis.

\vspace*{0.4cm}
This framework is the natural extension of Quantum Field Theory, which is constructed over a Minkowski background. Such a premise may be acceptable in current experimental research such as particle accelerators, but not for astrophysical and cosmological models, where the effects of gravity do not allow for such approximation. Therefore, classical dynamical backgrounds are considered, meaning that the matter fields are to be quantized but the gravitational field itself is not. It is expected that said transition will result in new effects to be observationally tested (although experiments might be out of our technological scope).

\vspace*{0.5cm}
After a formal introduction in the mathematical models that will be used throughout this thesis, we will present in each of the following chapters some new phenomenology particular to this area of Physics. The second chapter deals with the theory of scalar fields in expanding backgrounds; this cosmologically relevant scenario results in the breakage of what we understand as a particle, confronting Rabelais' \textit{Natura abhorret vacuum} with the realization of an infinite number of possible vacuum states, all equally valid. The third chapter explores the difference of vacuum measures between two non inertial observers, which yields a radiation capable of evaporating black holes. Finally, the last chapter will dive into the semiclassical approach of gravitational dynamics, explaining how quantum fields might affect gravity over cosmological scales. 

\vspace*{0.5cm}
\begin{comment}
	This thesis aimed at graduate students of theoretical physics, was written as part of the requirements for the award of a masters degree on advance physics, and thus assumes some prior knowledge on both Quantum Field Theory and General relativity; although an appendix regarding the basis of a free quantum scalar field in a flat background was added as a piece of reference on the matter. For a more in depth reference on both fields, one might want to look into common bibliographies such as \cite{BiblioQFT} and \cite{BiblioGR}.
\end{comment}